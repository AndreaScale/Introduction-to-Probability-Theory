\documentclass{book}
\usepackage[utf8]{inputenc}
\usepackage[english]{babel}
\usepackage{amsthm}
\usepackage{amsfonts}
\usepackage{subfiles}
\usepackage{graphicx}
\usepackage[scr]{rsfso}
\usepackage{hyperref}
\usepackage{mathtools}
\usepackage{array,amsfonts}
\usepackage{magaz}
\usepackage{amssymb}
\usepackage[shortlabels]{enumitem}
\usepackage{lipsum}
\usepackage[dottedtoc]{classicthesis}
\usepackage{tocloft}
\usepackage{cancel}

%-----------------------------------------------------------------------------------------------
\def\@tocline#1#2#3#4#5#6#7{\relax
  \ifnum #1>\c@tocdepth % then omit
  \else
    \par \addpenalty\@secpenalty\addvspace{#2}%
    \begingroup \hyphenpenalty\@M
    \@ifempty{#4}{%
      \@tempdima\csname r@tocindent\number#1\endcsname\relax
    }{%
      \@tempdima#4\relax
    }%
    \parindent\z@ \leftskip#3\relax \advance\leftskip\@tempdima\relax
    \rightskip\@pnumwidth plus4em \parfillskip-\@pnumwidth
    #5\leavevmode\hskip-\@tempdima #6\nobreak\relax
    \dotfill\hbox to\@pnumwidth{\@tocpagenum{#7}}\par
    \nobreak
    \endgroup
  \fi}
  
\newcommand\myfunc[5]{%
  \begingroup
  \setlength\arraycolsep{0pt}
  #1\colon\begin{array}[t]{c >{{}}c<{{}} c}
             #2 & \to & #3 \\ #4 & \to & #5 
          \end{array}%
  \endgroup}
\newtheorem{theorem}{Theorem}
\newtheorem{corollary}{Corollary}[theorem]
\newtheorem{lemma}[theorem]{Lemma}
\theoremstyle{remark}
\newtheorem{proposition}{Proposition}[section]
\theoremstyle{proposition}
\newtheorem*{remark}{Remark}
\theoremstyle{definition}
\newtheorem{definition}{Definition}[section]
\newtheorem{example}{Example}[section]
\newtheorem{observation}{Observation}[section]
\usepackage{tocloft}
\renewcommand{\cftchapdotsep}{\cftdotsep}




\begin{document}

\title{Appunti Calcolo delle Probabilità e Statistica}
\author{Andrea Scalenghe}
\date{Febbraio2021}
\begin{titlepage}
\maketitle
\end{titlepage}
\pdfbookmark[1]{\contentsname}{tableofcontents}
\setcounter{tocdepth}{2} % <-- 2 includes up to subsections in the ToC
\setcounter{secnumdepth}{3} % <-- 3 numbers up to subsubsections
\manualmark
\markboth{\spacedlowsmallcaps{\contentsname}}{\spacedlowsmallcaps{\contentsname}}
\tableofcontents
\automark[section]{chapter}
\renewcommand{\chaptermark}[1]{\markboth{\spacedlowsmallcaps{#1}}{\spacedlowsmallcaps{#1}}}
\renewcommand{\sectionmark}[1]{\markright{\thesection\enspace\spacedlowsmallcaps{#1}}}
\newpage
\part{Probabilità}
\chapter{Elementi fondamentali di probabilità}
\section{Introduzione}
Si introducono qui, in maniera intuitiva e non rigorosa, i concetti basilari di probabilità.
\theoremstyle{definition}
\begin{definition}
Un \textbf{esperimento probabilistico} è un fenomeno non deterministico e ripetibile
\end{definition}
\begin{example}
Data $f(x)$\(=e^x\), determinare l'immagine di un punto attraverso essa non è un esperimento probabilistico, in quanto deterministico (si conosce a priori l'esito dell'esperimento)
\end{example}

\begin{definition}
Uno \textbf{spazio campionario} è l'insieme dei possibili esiti di uno esperimento probabilistico. Si indica con \(\Omega\).
\end{definition}
\begin{example}
\hspace{2px}
\begin{itemize}
    \item Lo spazio campionario del lancio di una moneta è: \(\Omega=\{t,c\}\).
    \item Lo spazio campionario del lancio ripetuto di una moneta sino alla prima testa è: \(\Omega=\{t,ct,cct,ccct,...\}\).
    In questo caso \(|\Omega|=\mathbb{N}\).
\end{itemize}
\end{example}

Può essere utile rietichettare uno spazio campionario, mettendo i suoi elementi in corrispondeza con i numeri naturali.

\begin{definition}
Un \textbf{evento} è un un sottoinsieme dello spazio campionario
\end{definition}

\begin{example}
Nel lancio del dado bilanciato, dove \(\Omega=\{1,2,3,4,5,6\}\) è lo spazio campionario, tutti i seguenti sono eventi:
\begin{itemize}
    \item \(\{1\}\) uscita 1
    \item \(\{1,3,5\}\) uscita numero dispari (1 o 3 o 5)
    \item \(\Omega\) uscita di un numero
\end{itemize}
\end{example}

\begin{definition}
Gli elementi dello spazio campionario \(\Omega\) sono detti \textbf{eventi elementari}.
\end{definition}
\begin{definition}
Due eventi \(A,B\subseteq\Omega\) sono \textbf{incompatibili} sse \(A\cap B=\emptyset\)
\end{definition}

\begin{observation}
\(A,B\subseteq\Omega\) e \(B\subseteq\) A allora [A si verifica \(  \Rightarrow\) B si verifica]
\end{observation}

Vogliamo dare ora una definizione, che risulterà poi essere un caso particolare di quella generale, di probabilità.
\begin{definition}
Dato un esperimento probabilistico, i cui eventi elementari sono equiprobabili, sia \(\Omega\) il suo spazio campionario \textit{di dimensione finita} e sia \(A\subseteq\Omega\) un evento. La probabilità di A è:
\begin{center}
\(\mathbb{P}(A)=|A|/|\Omega|\)
\end{center}
\end{definition}
\newpage
\begin{observation}
\label{obs.1.2}
\begin{enumerate}
\hspace 1
    \item Se gli eventi sono incompatibili \(\mathbb{P}\) è addittiva:  \(\mathbb{P}(A\cup B)=\mathbb{P}(A)+\mathbb{P}(B)\)
    \item \(A\subseteq\Omega\) allora \(\mathbb{P}(A^c)=1-\mathbb{P}(A)\) infatti:
    \begin{center}
        \(1=\mathbb{P}(\Omega)=\mathbb{P}(A\cup A^c)=\mathbb{P}(A)+\mathbb{P}(A^c)\)
    \end{center}
    \item \(\mathbb{P}(\emptyset)=1-\mathbb{P}(\Omega)=0\)
    \item \(A,B\subseteq\Omega\) e \(A\subseteq B\) allora \(\mathbb{P}(A)\leq\mathbb{P}(B)\) infatti :
    \begin{center}
        \(B=A\cup (B\cap A^c)\) e \(A\cap (B\cap A^c)=\emptyset\), allora \(\mathbb{P}(B)=\mathbb{P}(A)+\mathbb{P}(B\cap A^c)\geq\mathbb{P}(A)\) ricordando che \(\mathbb{P}(C)\geq0\) \(\forall C\subseteq\Omega\)
    \end{center}
    \item \(\mathbb{P}(A)\geq0\) \(\forall A\subseteq\Omega\)
\end{enumerate}
\end{observation}
\vspace{30px}
Si può dornire una definizione geometrica di probabilità:

Presa una regione di piano G e g una porzione di essa, la probabilità che un punto di G cada in g è: 
\begin{center}
    \(\mathbb{P}("p\in g")=\)misura(g)/misura(G)
\end{center}
\newpage
\section{Calcolo Combinatorio}

Si farà ora un breve excursus su alcune delle fondamentali operazioni di calcolo combinatorio, che serviranno lungo tutto l'arco del corso.

Tratteremo in particolare le:
\begin{enumerate}
    \item Permutazioni
    \item Disposizioni
    \item Disposizioni con ripetizione 
    \item Combinazioni
    \item Combinazioni con ripetizione
    \label{enumerate:Permutazioni}
    \label{enumerate:Disposizioni}
    \label{enumerate:Disposizioni con ripetizione}
    \label{enumerate:Combinazioni}
    \label{enumerate:Combinazioni con ripetizione}
\end{enumerate}

\subsection{Permutazioni}
Sia (1,2,3,...,n) una sequenza ordinata di elementi. L'operazione di permutazione consiste nel riordinare gli elementi della sequenza.
\vspace{10px}
\newline
\textit{N.B.} La soprastante non ha alcuna itenzione di essere una definizione rigorosa, ma risulta sufficiente per gli scopi del corso.
\vspace{5px}
\newline
Il numero di possibili permutazioni di una sequenza di n numeri è: 
\begin{center}
    \scalebox{1.2}{\(\Pi_n=n!\)}
\end{center}

\subsection{Disposizioni}
Sia A un insieme di \(n\geq1\) elementi distinti. Si estraggono da A delle sequenze ordinate di lunghezza \(k\leq n\), senza possibilità di ripere gli elementi.

\begin{example}
\(A=\{ a,b,c \}\)  Dato il seguente insieme delle \textit{disposizioni} di lunghezza \(k=2\) sono per esempio: \((a,b)\) , \((c,b)\) , \((a,c)\)
\end{example}

Si denota con \(D_{n,k}=\#\) sequenze ordinate di \(k\) elementi scelti dagli \(n\) disponibili, e vale:
\begin{center}
    \scalebox{1.2}{\(D_{n,k}=\frac{n!}{(n-k)!}\)}
\end{center}

Questo risultato deriva dal fatto che vi siano \(k\) posti nei quali le opzioni di scelta degli elementi si riducono di uno sino al k-esimo posto. Si considera dunque il fattoriale di \(n\) troncato da \(n-k\).

Si può anche intendere \(D_{n,k}\) come il numero di permutazioni di \(n\) diviso il numero di permutazioni degli elementi non estratti.
\subsection{Disposizioni con ripetizione}
Sia A un insieme di \(n\geq1\) elementi distinti. Si estraggono da A delle sequenze ordinate, con possibilità di ripere gli elementi.

In questo caso non è detto che \(k\leq n\), si scrive quindi che: \(k\in \mathbb{N}^*\)\footnote{\(\mathbb{N}^*=\{ 1,2,3,...\}\) numeri naturali senza lo zero}.
\vspace{5px}
\begin{example}
\(A=\{ a,b\}\) e \(k=3\) alcune possibili combinazioni sono: \((a,a,a)\)  \((b,b,a)\) , \((a,a,b)\)
\end{example}
\vspace{10px}
Denotato con \(D'_{n,k}\) il numero di disposizioni con ripetizione di lunghezza k da n elementi, si ha:
\begin{center}
    \scalebox{1.2}{\(D'_{n,k}=n^k\)}
\end{center}
\vspace{10px}
\begin{observation}
Si può immaginare la disposizione \textit{senza} ripetizioni come un'estrazione di palline che vengono buttate una volta estratte, mentre la disposizione \textit{con} ripetizioni come un'estrazione di palline che vengono rimesse nel sacchetto una volta estratte.
\end{observation}

\subsection{Combinazioni}
Sia \(A\) un insieme di \(n\geq 1\) elementi distinti.
\newline
Le combinazioni di \(k\leq n\) elementi, non ripetibili, sono sottoinsiemi di \(A\), che \textit{non} tengono conto dell'ordine.

\vspace{10px}
Se si denota con \(C_{n,k}\) il numero di combinazioni di classe \(k\) (con \(k\) elementi) senza ripetizioni, si ha:
\begin{center}
    \scalebox{1.2}{\(C_{n,k}=\frac{n!}{k!(n-k)!}\)}
\end{center}

Il risultato ottenuto si evince dal numero di disposizioni di lunghezza k da n elementi possibili (\(D_{n,k}\)), dal quale va \textit{eliminato} l'ordine della sequenza, dividendo per il numero di permutazioni possibli della singola disposizione (\(\Pi_k\)).

\subsection{Combinazioni con ripetizione}
Sia \(A\) un insieme di \(n\geq 1\) elementi con \(k\in \mathbb{N}^*\). Come per le disposizioni la differenza sta nel poter riavere lo stesso elemento di \(A\) più volte.

\vspace{10px}
Denotato con \(C'_{n,k}\) il numero di combinazioni di \(n\) elementi di classe \(k\) con ripetizione, si ha:
\begin{center}
    \scalebox{1.2}{\(C'_{n,k}=\frac{(n+k-1)!}{k!(n-1)!}\)}
\end{center}
\newpage
\section{Assiomi della Probabilità}
\subsection{Definizione assiomatica di Probabilità}
Si deve la prima costruzione assiomatica della probabilità moderna ad A.N. Kolmogorov che nel 1933 pose le basi della teoria della probabilità. Egli stabilì gli assiomi che definiscono la probabilità di un dato evento in un esperimento probabilistico.

Iniziamo dando alcune definizioni preliminari.
\begin{definition}
Sia \(\Omega\) un insieme e sia $\mathscr{F}$ e una famiglia di sottoinsiemi di \(\Omega\), si dice che $\mathscr{F}$ è un'\textbf{algebra} se:
\begin{itemize}
    \item $\Omega\in\mathscr{F}$
    \item $\forall A \in \mathscr{F} \Rightarrow A^c \in \mathscr{F}$
    \item $A,B \in \mathscr{F} \Rightarrow A \cup B \in \mathscr{F}$          
\end{itemize}
\end{definition}
\noindent
Si osserva dunque che un'algebra è chiusa rispetto alla \textit{complementazione} e all'\textit{unione finita}.
\begin{definition}
Sia \(\Omega\) un insieme e sia $\mathscr{F}$ e una famiglia di sottoinsiemi di \(\Omega\), si dice che $\mathscr{F}$ è una \textbf{sigma-algebra} (\textbf{$\sigma$-algebra}) se:
\begin{itemize}
    \item $\Omega\in\mathscr{F}$
    \item $\forall A \in \mathscr{F} \Rightarrow A^c \in \mathscr{F}$
    \item $\{A_i\}_{i=1}^{+\infty}$ t.c. $A_i \in \mathscr{F} \Rightarrow \bigcup\limits_{i=1}^{+\infty} A_i \in \mathscr{F}$ 
\end{itemize}
\end{definition}
\noindent
Una \textit{$\sigma$-algebra} è dunque un'\textit{algebra} con chiusura rispetto all'unione \textit{numerabile}.

\begin{example}
Le seguenti sono $\sigma$-algebre:
\begin{itemize}
    \item $\mathscr{F} = \{\Omega , \emptyset\}$ Banale
    \item $\mathscr{F} = \mathscr{P}(\Omega)$ Discreta
\end{itemize}
\end{example}

\begin{proposition}
\label{prop.3.1}
$\{\mathscr{F}_i\}_{i\in I}$ con $\mathscr{F}_i$ $\sigma$-algebra $\forall$i $\Rightarrow \bigcap\limits_{i\in I}\mathscr{F}_i $ è una $\sigma$-algebra.   
\end{proposition}
La proposizione \ref{prop.3.1} ci permette di dare la seguente definizione.
\begin{definition}
Sia $\Omega$ un insieme e $\mathscr{C}$ una collezione di sottoinsiemi di $\Omega$. Siano $\mathscr{F}_i$ tutte le $\sigma$-algebre contenenti $\mathscr{C}$, si definisce:
\begin{center}
\begin{displaymath}
  \sigma(\mathscr{C})=\bigcap\limits^n_{i=1}\mathscr{F}_i
\end{displaymath}    
\end{center}
\`E la \textbf{$\sigma$-algebra generata} da $\mathscr{C}$.
\end{definition}

La $\sigma$-algebra generata da una famiglia di sottoinsiemi di $\Omega$ è unica, infatti se ve ne fossero due si includerebbero a vicenda, in quanto entrambe sarebbero le $\sigma$-algebre più piccole contenenti la famiglia di sottoinsiemi. 

\begin{definition}
Un \textbf{evento} è un elemento di una $\sigma$-algebra.
\end{definition}

\begin{definition}
Sia $\Omega$ un insieme e $\mathscr{F}$ una $\sigma$-algebra su $\Omega$. La coppia $(\Omega , \mathscr{F})$ è detta \textbf{spazio misurabile}.
\end{definition}

\begin{example}
Un esempio di \textit{spazio misurabile} è lo spazio campionario di Bernoulli dotato dell'insieme delle parti:
\begin{center}
    $\Omega=\{\omega=(\omega_1,\omega_2,\omega_3,...) | \omega_i \in \{0,1\}$ per $i\in\mathbb{N}^*\}$
\end{center}
\end{example}
\begin{example}
Un altro esempio è $\mathbb{R}$ con la $\sigma$-algebra dei \textbf{boreliani}, denotata con $\mathcal{B}(\mathbb{R})$. Dunque:
\begin{itemize}
    \item $\Omega=\mathbb{R}$
    \item $\mathscr{H}=\{(a,b)\subseteq\mathbb{R}$ $|$ $a<b \}$
\end{itemize}
La $\sigma$-algebra dei boreliani è definibile su ogni spazio topologico, prendendo tutti gli aperti di esso come famiglia di sottoinsiemi.

\begin{observation}
Se si prende $\mathscr{H}'=\{(-\infty$, a ] , a $\in\mathbb{R}\}$ allora $\sigma(\mathscr{H})=B(\mathbb{R})$, infatti:
\newline
$\subseteq$ : $(-\infty, a]=(a, b)^c\in B(\mathbb{R})$
\newline
$\supseteq$ : $(a, b)=(-\infty , a]^c\cap(-\infty , b)=\{(-\infty , a]^c\cap(\bigcup\limits_{i=1}^{+\infty}(-\infty , b_i])\}\in\sigma(\mathscr{H}')$ 

Prendendo una successione tale che $b_i\rightarrow b$.
\end{observation}
\end{example}
\vspace{10px} 
Possiamo ora, finalmente, definire la \textit{misura di probabilità} di un evento.
\begin{definition}
Dato uno spazio misurabile $(\Omega , \mathscr{F})$, una funzione d'insieme
\newline
$\mathbb{P}:\mathscr{F}\longrightarrow\mathbb{R}$ tale che:
\begin{itemize}
    \item $\forall A\in\mathscr{F} , \mathbb{P}(A)\geq0$
    \item $\mathbb{P}(\Omega)=1$
    \item $\{A_n\}_{n\in\mathbb{N}}\subseteq\mathscr{F}$ $|$ $A_i\cap A_j=\emptyset$  $\forall i\neq j \Rightarrow \mathbb{P}(\bigcup\limits^{+\infty}_{i=1}A_i)=\sum\limits_{i=1}^{+\infty}\mathbb{P}(A_i)$
\end{itemize}
\`E detta \textbf{misura di probabilità}.
\end{definition}

Si osserva che valgono le propietà dell'osservazione \ref{obs.1.2}, seppur derivate leggermente differentemente.
\vspace{5px}

E inoltre vale la proprietà: 
\begin{equation}
    \mathbb{P}(A\cup B)=\mathbb{P}(A)+\mathbb{P}(B)-\mathbb{P}(A\cap B)
    \label{Base_ind_T1}
\end{equation} 


La cui dimostrazione è lasciata al lettore. (zero sbatta oggi)

\vspace{10px}
\begin{theorem}[Principio di inclusione-esclusione]
Sia $(\Omega,\mathscr{F},\mathbb{P})$ uno spazio \newline probabilistico e siano $A_1,A_2,...,A_n$ eventi:
\begin{center}
    $\mathbb{P}(\bigcup\limits\limits_{i=1}^n A_i)= \sum\limits_{i=1}^n \mathbb{P}(A_i) - \sum\limits_{i\leq j}^n\mathbb{P}(A_i\cap A_j) + \sum\limits_{i\leq j\leq k}^n\mathbb{P}(A_i\cap A_j\cap A_k) - ...$
    \newline
    $... (-1)^{n+1}\mathbb{P}(\bigcap\limits_{i=i}^nA_i)$
\end{center}
\newpage
\begin{proof}
Si procede per induzione su n.
\newline
Base induzione: è \ref{Base_ind_T1}.
\newline
Passo induttivo: Assumiamo vera l'implicazione per n. Applicando la propietà \ref{Base_ind_T1} ottengo:
\begin{center}
            $\mathbb{P}(\bigcup\limits\limits_{i=1}^{n+1} A_i)=\mathbb{P}(\bigcup\limits\limits_{i=1}^n A_i \cup A_{n+1})=\mathbb{P}(\bigcup\limits_{i=1}^nA_i)+\mathbb{P}(A_{n+1})-\mathbb{P}((\bigcup\limits_{i=1}^n A_i) \cap (A_{n+1}))$
\end{center}
\label{T1_EQ_1}
Osservo ora che al primo addendo posso applicare l'ipotesi induttiva. Mentre rispetto all'ultimo addendo:
\begin{center}
    $\mathbb{P}((\bigcup\limits_{i=1}^n A_i) \cap (A_{n+1}))=\mathbb{P}(\bigcup\limits_{i=1}^n (A_i \cap A_{n+1}))$
\end{center}
Applicando nuovamente l'ipotesi induttiva:
\begin{center}
    $\mathbb{P}(\bigcup\limits_{i=1}^n (A_i \cap A_{n+1}))=\sum\limits_{i=1}^n \mathbb{P}(A_i\cap A_{n+1}) - \sum\limits_{i\leq j}^n\mathbb{P}(A_i\cap A_j\cap A_{n+1}) +$
    \newline
    $+ \sum\limits_{i\leq j\leq k}^n\mathbb{P}(A_i\cap A_j\cap A_k\cap A_{n+1}) - ...  (-1)^{n+1}\mathbb{P}(\bigcap\limits_{i=i}^{n+1}A_i)$
\end{center}
Dunque osservando che:
\begin{center}
    $\sum\limits_{i\leq j}^{n+1}\mathbb{P}(A_i\cap A_j)=\sum\limits_{i\leq j}^n\mathbb{P}(A_i\cap A_j)+\sum\limits_{i=1}^n\mathbb{P}(A_i\cap A_{n+1})$
\end{center}
E questo vale per tutti gli addendi successivi.

Sostituendo nella prima equazione e raggruppando le sommatorie ottengo la tesi.
\end{proof}
\end{theorem}

\begin{theorem}[Disuguaglianza di Boole]
    Sia $(\Omega,\mathscr{F},\mathbb{P})$ uno spazio misurabile e $\{A_n\}_{n=1}^{+\infty}$ una successione di eventi, allora:
    \begin{center}
        $\mathbb{P}(\bigcup\limits_{n=1}^{+\infty}A_n) \leq \sum\limits_{n=1}^{+\infty}\mathbb{P}(A_n)$
    \end{center}
\begin{proof}
$\bigcup\limits_{n=1}^{+\infty}A_n=A_1\cup(A_1^c\cap A_2)\cup(A_1^c\cap A_2^c\cap A_3)\cup ...$

Si osserva che è un'unione disgiunta, e grazie alla monotonia di $\mathbb{P}$ posso concludere:
\begin{center}
    $\mathbb{P}(\bigcup\limits_{n=1}^{+\infty}A_n)=\mathbb{P}(A_1)+\mathbb{P}(A_1^c\cap A_2)+\mathbb{P}(A_1^c\cap A_2^c\cap A_3)+...\leq \sum\limits_{n=1}^{+\infty}A_n$
\end{center}
\end{proof}
\end{theorem}

\newpage    
\subsection{Continuità di $\mathbb{P}$}
\begin{definition}
Sia $\{\Omega, \mathscr{F}\}$ uno spazio misurabile e $\Pi:\mathscr{F}\longrightarrow\mathbb{R}$ una funzione, si dice che $\Pi$ è \textbf{continua} se:
\begin{center}
    $\lim_{x\to+\infty} \Pi(A_n)=\Pi(\lim_{x\to\infty} A_n)$
\end{center}
\end{definition}

Si vuole ora però definire rigoroamente cosa s'intende per limite di una successione di sottoinsiemi: $\lim_{x\to\infty} A_n$.

\begin{definition}
Sia $(C_n)_{n\in \mathbb{N}}$ una successione di insiemi, si dice:
\begin{itemize}
    \item \textbf{Crescente} se $C_n\subseteq C_{n+1}$ $\forall n\in\mathbb{N}$ e si definisce:
    \begin{center}
        $\lim_{x\to\infty} C_n=\bigcup\limits_{n=1}^{+\infty}C_n$
    \end{center}
    \item \textbf{Decrescente} se $C_n\supseteq C_{n+1}$ $\forall n\in\mathbb{N}$ e si definisce:
    \begin{center}
        $\lim_{x\to\infty} C_n=\bigcap\limits_{n=1}^{+\infty}C_n$
    \end{center}
\end{itemize}
\end{definition}

Si definiscono inoltre i limiti superiore e inferiore di una successione di insemi.

\begin{definition}
    Sia $(A_n)_{n\in \mathbb{N}}$ una successione di insiemi, si definiscono:
\begin{itemize}
    \item $\liminf_{x\to\infty} A_n=\{\omega\in\Omega \mid \exists\hat{n}\in\mathbb{N}$ t.c. $\forall k\geq\hat{n}$ 
    $\omega\in A_k\}$
    \item $\limsup_{x\to\infty} A_n=\{\omega\in\Omega \mid \forall n\in\mathbb{N}$  $\exists k\geq n$ 
    $\omega\in A_k\}$
\end{itemize}

Dunque si dice che $(A_n)_{n\in \mathbb{N}}$ \textbf{ammette limite} se:
\begin{center}
    $\limsup_{x\to\infty} A_n=\liminf_{x\to\infty} A_n$
\end{center}
\end{definition}

Due definzioni equivalenti ma utili dal punto di vista del calcolo, e che forse rendono anche più chiari i concetti sono le seguenti:
\begin{itemize}
    \item $\liminf_{x\to\infty} A_n= \large\bigcup\limits_{n=1}^{+\infty} \bigcap\limits_{k\geq n} A_n=\lim_{x\to\infty} C_n$
    \item $\limsup_{x\to\infty} A_n= \large\bigcap\limits_{n=1}^{+\infty} \bigcup\limits_{k\geq n} A_n=\lim_{x\to\infty} D_n$
\end{itemize}

Dove $(D_n)_{n\in\mathbb{N}}=\bigcup\limits_{k\geq n} A_n$ e $(C_n)_{n\in\mathbb{N}}=\bigcap\limits_{k\geq n} A_n$

\begin{example}
    $\Omega=[-1,1]$ , $\mathscr{F}=\mathscr{P}(\Omega)$ e sia $(A_n)_{n\in\mathbb{N}} = (0,1)$ se n è  \textbf{pari} e $(A_n)_{n\in\mathbb{N}} = (0,\frac{1}{n})$ se n è  \textbf{dipari}. Si ottiene:
    \begin{itemize}
        \item $\limsup_{x\to\infty} A_n= (0,1)$
        \item $\liminf_{x\to\infty} A_n= \emptyset$
    \end{itemize}
    Dunque $(A_n)_{n\in\mathbb{N}}$ \textbf{non} ammette limite.
\end{example}
\newpage
Contrariamente si osserva una successione \textit{dotata} di limite nel seguente esempio.
\begin{example}
     $\Omega=[-1,1]$ , $\mathscr{F}=\mathscr{P}(\Omega)$ e sia $(A_n)_{n\in\mathbb{N}} = \emptyset$ se n è  \textbf{pari} e $(A_n)_{n\in\mathbb{N}} = (0,\frac{1}{n})$ se n è  \textbf{dipari}. Si ottiene:
    \begin{itemize}
        \item $\limsup_{x\to\infty} A_n= \emptyset$
        \item $\liminf_{x\to\infty} A_n= \emptyset$
    \end{itemize}
    Dunque $(A_n)_{n\in\mathbb{N}}$ \textbf{ammette} limite.
\end{example}

Si enuncia ora un importante teorema che lega la $\sigma$-additività alla continuità di $\mathbb{P}$.

\begin{theorem}
   Sia $(\Omega,\mathscr{F})$ uno spazio misurabile e $\mathbb{P}:\mathscr{F}\longrightarrow\mathbb{R}$ una funzione tale che:
   \begin{itemize}
       \item $\mathbb{P}(\Omega)=1$
       \item $\mathbb{P}(A)\geq0$
   \end{itemize}
   Vale:
   \begin{center}
       $\mathbb{P}$ è $\sigma$-additiva $\iff$ $\mathbb{P}$ è continua e finito additiva
   \end{center}
   
\begin{proof}
$\Rightarrow)$ Divido questa implicazione in 4 passi:
\begin{enumerate}
\item Sia $(C_n)_{n\in\mathbb{N}}\subseteq\Omega$ successione decrescente a $\emptyset$, posso dunque riscrivere:
\begin{center}
    $C_n=\bigcup\limits_{k=n}^{+\infty}(C_k\cap C_{k+1}^c)$
\end{center}
Così facendo osservo che, grazie alla $\sigma$-additività:
\begin{center}
    $\sum\limits_{k=1}^{+\infty}(C_k\cap C_{k+1}^c)=\mathbb{P}(\bigcup\limits_{k=1}^{+\infty}(C_k\cap C_{k+1}^c))=\mathbb{P}(C_1)\leq1$
\end{center}
 Dunque $\sum\limits_{k=1}^{+\infty}(C_k\cap C_{k+1}^c)$ converge e:
\begin{center}
    $\lim_{n\to+\infty}\mathbb{P}(C_n)=\lim_{n\to+\infty}\mathbb{P}(\bigcup\limits_{k=n}^{+\infty}(C_k\cap C_{k+1}^c))=$
    \newline
    $\lim_{n\to+\infty}\sum\limits_{k=n}^{+\infty}\mathbb{P}(C_k\cap C_{k+1}^c)=\lim_{n\to+\infty}\sum\limits_{k=1}^{+\infty}\mathbb{P}(C_k\cap C_{k+1}^c)-\sum\limits_{k=1}^{n-1}\mathbb{P}(C_k\cap C_{k+1}^c)=0$
\end{center}
Ottenendo così la continuità di $\mathbb{P}$, con limite tendente a zero, rispetto a successioni decrescenti all'inseme vuoto.
\label{T3_1}
\item Sia $(D_n)_{n\in\mathbb{N}}\subseteq\Omega$ successione decrescente a $D$, posso dunque riscrivere:
\begin{center}
    $C_n=D_n\cap D^c$ cioè $D_n=C_n\cup D$
\end{center}
Ottenendo $(C_n)$ decrescente all'insieme vuoto e $(D_n)$ riscritta in funzione di insiemi disgiunti. Usando \ref{T3_1}:
\begin{center}
    $\lim_{n\to+\infty}\mathbb{P}(D_n)=\lim_{n\to+\infty}\mathbb{P}(C_n\cup D)=\lim_{n\to+\infty}\mathbb{P}(C_n)+\mathbb{P}(D)=$
\vspace{7px}    
\newline
    $=\lim_{n\to+\infty}\mathbb{P}(D)=\mathbb{P}(\lim_{n\to+\infty}D_n)$
\end{center}
Ottendendo la continuità di $\mathbb{P}$ per una successione decrescente.
\item Sia $(B_n)_{n\in\mathbb{N}}\subseteq\Omega$ successione crescente a $B$. Osservo che $(B_n^c)_{n\in\mathbb{N}}$ è una successione decrescente a $B^c$, quindi:
\begin{center}
         $\lim_{n\to+\infty}\mathbb{P}(B_n)=\lim_{n\to+\infty}(1-\mathbb{P}(B_n^c))=1-\lim_{n\to+\infty}\mathbb{P}(B_n^c)=$
         \vspace{7px}
         \newline
         $1-\mathbb{P}(B^c)=\mathbb{P}(B)=\mathbb{P}(\lim_{n\to+\infty}B_n)$
\end{center}
Ottenendo la continuità di $\mathbb{P}$ per una successione crescente.
\item Sia $(A_n)_{n\in\mathbb{N}}$ una successione che ammette limite $A$, vale:
\begin{center}
    $\bigcap\limits_{k\geq n}A_k\subseteq A_n\subseteq\bigcup\limits_{k\geq n}A_k$
\end{center}
Sfruttando la monotonia di $\mathbb{P}$:
\begin{center}
    $\mathbb{P}(\bigcap\limits_{k\geq n}A_k)\leq\mathbb{P}(A_n)\leq\mathbb{P}(\bigcup\limits_{k\geq n}A_k)$
\end{center}
Osservo che $(\bigcap\limits_{k\geq n}A_k)$ e $(\bigcup\limits_{k\geq n}A_k)$ sono due successioni rispettivamente decrescenti e crescenti. Passando al limite:
\begin{center}
    $\lim_{n\to +\infty}\mathbb{P}(\bigcap\limits_{k\geq n}A_k)\leq\lim_{n\to +\infty}\mathbb{P}(A_n)\leq\lim_{n\to +\infty}\mathbb{P}(\bigcup\limits_{k\geq n}A_k)$
\end{center}
\begin{center}
    $\mathbb{P}(\lim_{n\to +\infty}\bigcap\limits_{k\geq n}A_n)\leq\lim_{n\to +\infty}\mathbb{P}(A_n)\leq\mathbb{P}(\lim_{n\to +\infty}\bigcup\limits_{k\geq n}A_n)$
\end{center}
\begin{center}
    $\mathbb{P}(\liminf_{n\to +\infty}A_n)\leq\lim_{n\to +\infty}\mathbb{P}(A_n)\leq\mathbb{P}(\limsup_{n\to +\infty}A_n)$
\end{center}
Poichè $(A_n)$ è dotata di limite: $\liminf_{n\to +\infty}A_n=\limsup_{n\to +\infty}A_n=A$.
Infine si ottiene:
\begin{center}
    $\lim_{n\to +\infty}\mathbb{P}(A_n)=\mathbb{P}(lim_{n\to +\infty}A_n)$
\end{center}
\end{enumerate}
$\Leftarrow)$ Sia $(A_n)_{n\in\mathbb{N}}$ una successione tale che $A_i\cap A_j=\emptyset$ $\forall i\neq j$. Usando la finito additività:
\begin{center}
    $\mathbb{P}(\bigcup\limits_{n=1}^{+\infty}A_n)=\mathbb{P}(\bigcup\limits_{n=1}^{k-1}A_n\cup\bigcup\limits_{n=k}^{+\infty}A_n)=\sum\limits_{n=1}^{k-1}\mathbb{P}(A_n) + \mathbb{P}(\bigcup\limits_{n=k}^{+\infty}A_n)$
\end{center}
Osservo ora che $C_k=\bigcup\limits_{n=k}^{+\infty}A_n$ è una successione decrescente a $\emptyset$, poichè grazie alla disgiunzione dei suo elementi:
\begin{center}
    $\forall\omega\in C_1$ $\exists\hat{k}$ t.c. $\omega\in C_{\hat{k}}$ $\Rightarrow$ $\omega\notin C_h$ $h\geq\hat{k}$
\end{center}
Dunque passando al limite e sfruttando la continuità di $\mathbb{P}$:
\begin{center}
    $\lim_{k\to+\infty}\mathbb{P}(\bigcup\limits_{n=1}^{+\infty}A_n)=\lim_{k\to+\infty}(\sum\limits_{n=1}^{k}\mathbb{P}(A_n) + \mathbb{P}(\bigcup\limits_{n=k}^{+\infty}A_n))=$
    \vspace{7px}
    \newline
    $\sum\limits_{n=1}^{+\infty}\mathbb{P}(A_n) + \mathbb{P}(\lim_{k\to+\infty}(\bigcup\limits_{n=k}^{+\infty}A_n))=\sum\limits_{n=1}^{+\infty}\mathbb{P}(A_n) + \mathbb{P}(\lim_{k\to+\infty}C_n))=\sum\limits_{n=1}^{+\infty}\mathbb{P}(A_n)$
\end{center}
Si conclude così la $\sigma$-additività di $\mathbb{P}$.
\end{proof}
\end{theorem}
\newpage
\subsection{Dipendenza, Indipendenza e Condizionamento}
Si introducono qui i concetti di dipendenza, indipendenza e condizionamento.

\begin{definition}
    Sia $(\Omega,\mathscr{F},\mathbb{P})$ uno spazio misurabile e $A,B\in\Omega$ due eventi, si dice che:
    \begin{itemize}
        \item $A$ e $B$ sono \textbf{indipendenti} se: $\mathbb{P}(A\cap B)=\mathbb{P}(A)\mathbb{P}(B)$
        \item $\Omega$ è un evento \textbf{certo}
        \item $\emptyset$ è un evento \textbf{impossibile}
        \item $A$ è un evento \textbf{quasi-certo} se: $\mathbb{P}(A)=1$
        \item $B$ è un evento \textbf{quasi-impossibile} se: $\mathbb{P}(B)=0$
    \end{itemize}
\end{definition}

\begin{proposition}
    Sia $A\in\mathscr{F}$ un evento, $A$ è indipendente da ogni evento quasi-certo e da ogni evento quasi-impossibile.
    
\noindent
\begin{proof}
\begin{center}
    $\mathbb{P}(A\cap\Omega)=\mathbb{P}(A)=\mathbb{P}(A)\mathbb{P}(\Omega)$
\end{center}
\begin{center}
    $\mathbb{P}(A\cap\emptyset)=\mathbb{P}(\emptyset)=0=\mathbb{P}(A)\mathbb{P}(\Omega)$
\end{center}
Sia $B$ un evento quasi-certo: $A=(A\cap B)\cup(A\cap B^c)$
\begin{center}
    $\mathbb{P}(A)=\mathbb{P}((A\cap B)\cup(A\cap B^c))=\mathbb{P}(A\cap B) + \mathbb{P}(A\cap B^c)$
\end{center}
Osservando che per monotonia di $\mathbb{P}$:
\begin{center}
    $0\leq\mathbb{\emptyset}\leq\mathbb{P}(A\cap B^c)\leq\mathbb{P}(B^c)=1-\mathbb{P}(B)=0$
\end{center}
Rimane:
\begin{center}
    $\mathbb{P}(A\cap B)=\mathbb{P}(A)\mathbb{P}(B)$
\end{center}

Analogamente, riscrivendo $C=(C\cap A)\cup(C\cap A^c)$, si dimostra per i quasi-impossibili.
\end{proof}
\end{proposition}

\begin{proposition}
    $A,B\in\mathscr{F}$ eventi indipendenti allora lo sono anche $(A,B^c)$ , $(A^c,B)$ , $(A^c,B^c)$.
\end{proposition}

Dimostrazione lasciata al lettore. (Usare $A=(A\cap B)\cup(A\cap B^c)$)


\begin{definition}[Pairwise independence]
    Sia $(\Omega,\mathscr{F},\mathbb{P})$ uno spazio di probaibliltà e $\{{A_i}\}_{i=1}^{+\infty}$ una collezione numerabile di eventi. Si dice che gli eventi di $\{{A_i}\}_{i=1}^{+\infty}$ sono \textbf{due a due indipendenti} se:
    \begin{center}
        $\mathbb{P}(A_i\cap A_j)=\mathbb{P}(A_i)\mathbb{P}(A_j)$ $\forall i\neq j$
    \end{center}
\end{definition}

\begin{definition}[Mutual independence]
    Sia $(\Omega,\mathscr{F},\mathbb{P})$ uno spazio di probaibliltà e $\{{A_i}\}_{i=1}^{+\infty}$ una collezione numerabile di eventi. Si dice che gli eventi di $\{{A_i}\}_{i=1}^{+\infty}$ sono \textbf{mutuamente indipendenti} se:
    \begin{center}
        $\mathbb{P}(A_{i,1}\cap A_{i,2}\cap ...\cap A_{i,k})=\mathbb{P}(A_{i,1})\mathbb{P}(A_{i,2})...\mathbb{P}(A_{i,k})$  $\forall k\in\mathbb{N}$ , $\forall(i_1,...,i_k) $
    \end{center}
\end{definition}

\begin{observation}
    Si osserva immediatamente che la mutua indipendenza implica l'indipendenza due a due.
\end{observation}
\newpage
Non vale però l'inverso (le definizioni non sono equivalenti). Se per esempio si prende un'urna con tre biglietti: uno con il numero 1, uno con il numero 2 e uno con il numero 1 e 2. Posto $A_i$ come l'evento "uscita del numero i" allora:
\begin{center}
    $\mathbb{P}(A_1\cap A_2)=\frac{1}{3}$   mentre    $\mathbb{P}(A_1)\mathbb{P}(A_2)=\frac{1}{6}$
\end{center}
Dunque gli eventi \textit{non} sono mutuamente indipendenti, ma si dimostra essere indipendenti due a due.

\begin{definition}
    Sia $(\Omega,\mathscr{F},\mathbb{P})$ uno spazio di probaibliltà e $A,B\in\mathscr{F}$ t.c. $\mathbb{P}(B)>0$ (non nulla). Si definisce la \textbf{probabilità di $A$ condizionata a $B$} come:
    \begin{center}
        $\mathbb{P}(A | B)=\mathbb{P}_B(A)=$\large$\frac{\mathbb{P}(A\cap B)}{\mathbb{P}(B)}$
    \end{center}
\end{definition}

\begin{proposition}
    Sia $(\Omega,\mathscr{F},\mathbb{P})$ uno spazio di probaibliltà e $A,B\in\mathscr{F}$ t.c. $\mathbb{P}(B),\mathbb{P}(A)>0$, si equivalgono:
    \begin{enumerate}
        \item $\mathbb{P}(A\cap B)=\mathbb{P}(A)\mathbb{P}(B)$
        \item $\mathbb{P}_B(A)=\mathbb{P}(A)$
        \item $\mathbb{P}_A(B)=\mathbb{P}(B)$
    \end{enumerate}
    \begin{proof}
    
La dimostrazione è banale e lasciata al lettore.
    \end{proof}
\end{proposition}

\vspace{5px}

\begin{proposition}
    Sia $(\Omega,\mathscr{F},\mathbb{P})$ uno spazio di probaibliltà e $B\in\mathscr{F}$ t.c. 
    \newline
    $\mathbb{P}(B)>0$, allora :
    \begin{center}
        $\mathbb{P}_B(\cdot)$ è una misura di probabilità.
    \end{center}
    \vspace{5px}
\begin{proof}
\begin{enumerate}
    \item $\mathbb{P}_B(\Omega)=${\large$\frac{\mathbb{P}(\Omega\cap B)}{\mathbb{P}(B)}=\frac{\mathbb{P}(B)}{\mathbb{P}(B)}=1$}
    \item $A\in\mathscr{F}$ dunque: $\mathbb{P}_B(A)=${\large$\frac{\mathbb{P}(A\cap B)}{\mathbb{P}(B)}$}$\geq0$ in quanto rapporto di quantità non negative (per def. di $\mathbb{P}$)
    \item $\{{A_i}\}_{i=1}^{+\infty}$ una successione di eventi disgiunti: 
    \begin{center}
        $\mathbb{P}_B(\bigcup\limits_{i=1}^{+\infty}{A_i})=${\large$\frac{\mathbb{P}(\bigcup\limits_{i=1}^{+\infty}{A_i}\cap B)}{\mathbb{P}(B)}=\frac{\sum\limits_{i=1}^{+\infty}\mathbb{P}(A_i\cap B)}{\mathbb{P}(B)}$}$=\sum\limits_{i=1}^{+\infty}\mathbb{P}_B(A_i)$
    \end{center}
\end{enumerate}
\end{proof}    
\end{proposition}
\vspace{5px}

\begin{theorem}[Formula delle probabilità totali]
    Sia $(\Omega,\mathscr{F},\mathbb{P})$ uno spazio di probaibliltà e $\{{A_i}\}_{i=1}^{+\infty}$ una partizione di $\Omega$ t.c. $\mathbb{P}(A_i)\geq0$ $\forall i$, allora vale:
    \begin{center}
        $\mathbb{P}(A)=\sum\limits_{i=1}^{+\infty}\mathbb{P}_{A_i}(A)\mathbb{P}(A_i)$ , $\forall A\in\mathscr{F}$
    \end{center}
\begin{proof}
\begin{center}
$\mathbb{P}(A)=\mathbb{P}(A\cap \Omega)=\mathbb{P}(A\cap\bigcup\limits_{i=1}^{+\infty}A_i)=\mathbb{P}(\bigcup\limits_{i=1}^{+\infty}A\cap A_i)=\sum\limits_{i=1}^{+\infty}\mathbb{P}(A\cap A_i)=\sum\limits_{i=1}^{+\infty}\mathbb{P}_{A_i}(A)\mathbb{P}(A_i)$
\end{center}
\end{proof}    
\end{theorem}

\begin{theorem}[Teorema di Bayes]
    Sia $(\Omega,\mathscr{F},\mathbb{P})$ uno spazio di probaibliltà, $\{{A_i}\}_{i=1}^{+\infty}$ una partizione di $\Omega$ e $B\in\mathscr{F}$ vale:
    \begin{center}
        $\mathbb{P}_B(A_i)=${\large$\frac{\mathbb{P}_{A_i}(B)\mathbb{P}(A_i)}{\sum\limits_{i=1}^{+\infty}\mathbb{P}_{A_i}(B)\mathbb{P}(A_i)}$}
    \end{center}
\begin{proof}
\vspace{5px}
    \[\mathbb{P}_B(A_i)={\large\frac{\mathbb{P}(A_i\cap B)}{\mathbb{P}(B)}=\frac{\mathbb{P}_{A_i}(B)\mathbb{P}(A_i)}{\sum\limits_{i=1}^{+\infty}\mathbb{P}_{A_i}(B)\mathbb{P}(A_i)}}\]
\end{proof}    
\end{theorem}
\newpage
\chapter{Variabili aleatorie}
\begin{definition}
    Siano $(A,\mathcal{A})$ e $(E,\mathcal{E})$ due spazi misurabili, una funzione $f:\mathcal{A}\longrightarrow\mathcal{E}$ è detta \textbf{misurabile} se:
    \begin{center}
         $B\in\mathcal{E} \Rightarrow f^{-1}(B)\in\mathcal{A}$ \hspace{4px} $\forall B\in\mathcal{E}$ 
    \end{center}
\end{definition}

\vspace{5px}

\begin{definition}
    Sia $(\Omega,\mathscr{F},\mathbb{P})$ uno spazio probabilistico e $(E,\mathcal{E})$ uno spazio misurabile, una funzione $X:\Omega\longrightarrow\mathcal{E}$ \textit{misurabile} è una \textbf{variabile aleatoria}.
\end{definition}

\vspace{10px}

\begin{example}
    Si prende come spazio misurabile $(\mathbb{R},\mathcal{B}(\mathbb{R}))$ (i reali con i suoi boreliani). 
    \begin{itemize}
        \item $\mathbb{P}(X=3)=\mathbb{P}(X^{-1}(\{3\}))$
        \item $\mathbb{P}(X\geq7)=\mathbb{P}(X^{-1}([7,+\infty)))$
        \item $\mathbb{P}(X^{-1}(B))=\mathbb{P}(\{\omega\in\Omega$ $|$ $X(\omega)\in B\})$ \hspace{4px} $\forall B\in\mathcal{B}(\mathbb{R})$
    \end{itemize}
\end{example}

\vspace{10px}

\begin{theorem}
    Sia $(\Omega,\mathscr{F},\mathbb{P})$ uno spazio probabilistico, $(E,\mathcal{E})$ uno spazio
    \newline
    misurabile e $X:\Omega\longrightarrow\mathcal{E}$ una variabile aleatoria, allora:
    \begin{center}
        $\mathbb{P}_X(\cdot)\coloneqq\mathbb{P}(X^{-1}(\cdot))$ al variare di $B\in\mathcal{E}$ 
    \end{center}
    è una \textbf{misura di probabilità}.
\end{theorem}

\vspace{5px}

\begin{itemize}
    \item $\mathbb{P}_X(\cdot)$ è detta \textbf{distribuzione di X} oppure \textbf{legge di X}
    \item Variabili aleatorie (V.A.) diverse che inducono la stessa distribuzione si dicono \textbf{identicamente distribuite}
\end{itemize}

\vspace{10px}

\begin{definition}
    Data $X$ una variabile aleatoria in $\mathbb{R}$, la funzione
    \newline 
    $F_X:\mathbb{R}\longrightarrow[0,1]$ t.c. $F_X(x)=\mathbb{P}(X\leq x)$ si dice \textbf{funzione di distribuzione}.
\end{definition}



Il seguente teorema ci fornirà importanti informazioni sulla natura della funzione di distribuzione.

\begin{theorem}
    Data una funzione di distribuzione $F_X$ valgono le seguenti:
    \begin{enumerate}
        \item $F_X$ è non decrescente
        \item $F_X$ è continua da destra
        \item  $\lim_{x\to+\infty} F_X(x) = 1$ e $\lim_{x\to-\infty} F_X(x) = 0$
    \end{enumerate}
\begin{proof}
\begin{enumerate}
    \item Siano $x,y\in\mathbb{R}$ t.c. $x\leq y$ allora:
    \begin{center}
        $\{\omega\in\Omega$ $|$ $X(\omega)\leq x\} \subseteq \{\omega\in\Omega$ $|$ $X(\omega)\leq y\}$
    \end{center}
    Dunque per monotonia di $\mathbb{P}$ si ha:
    \begin{center}
        $\mathbb{P}(X\leq x)\leq\mathbb{P}(X\leq y)$
        \hspace{4px}
        $\Rightarrow$
        \hspace{4px}
        $F_X(x)\leq F_X(y)$
    \end{center}
    \item Voglio mostrare $\lim_{x\to x_0^+} F_X(x) = F_X(x_0)$:
    \vspace{10px}
    \newline
    \noindent
    Considero $A_n=\{\omega\in\Omega$ $|$ $X(\omega)\in[x_0,x_0+\frac{1}{n})\}$ al variare di $n\in\mathbb{N}$ e osservo essere una successione decrescente all'insieme vuoto. 
    \vspace{10px}
    \newline
    \noindent
    Dunque applicando la definizione di funzione di probabilità:
    \begin{center}
        $\lim_{x\to x_0^+} F_X(x)=\lim_{n\to+\infty} F_X(x_0+\frac{1}{n})=\lim_{n\to+\infty}\mathbb{P}(X\leq x_0+\frac{1}{n})=$
        \vspace{5px}
        \newline
        $=\lim_{n\to+\infty}\mathbb{P}(\{\omega\in\Omega$ $|$ $X(\omega)\in(-\infty,x_0+\frac{1}{n})\})=$
        \vspace{5px}
        \newline
        $=\lim_{n\to+\infty}\mathbb{P}(\{\omega\in\Omega$ $|$ $X(\omega)\in(-\infty,x_0)\}\cup(A_n))=$
        \vspace{5px}
        \newline
        $=\mathbb{P}(X\leq x_0)+\lim_{n\to+\infty}\mathbb{P}(A_n)=F_X(x_0)$
    \end{center}
    \item 
    \begin{itemize}
        \item $\lim_{x\to+\infty} F_X(x)=\lim_{x\to+\infty}\mathbb{P}(\{\omega\in\Omega$ $|$ $X(\omega)\in(-\infty,x)\})=\mathbb{P}(\Omega)=1$
        \item $\lim_{x\to-\infty} F_X(x)=\lim_{x\to-\infty}\mathbb{P}(\{\omega\in\Omega$ $|$ $X(\omega)\in(-\infty,x)\})=\mathbb{P}(\emptyset)=0$
    \end{itemize}
\end{enumerate}
\end{proof}    
\end{theorem}



\begin{proposition}
Sia $X$ una variabile indipendente a valori in $\mathbb{R}$ e siano
\newline
$a,b\in\mathbb{R}$ tali che $a\leq b$, allora valgono:
\begin{enumerate}
    \item $\mathbb{P}(a< X\leq b) = F_X(b)-F_X(a) $
    \item $\mathbb{P}(X = a) = F_X(a) - \lim_{x\to a^-}F_X(x)$
    \item $\mathbb{P}(a< X< b) = \lim_{x\to b^-}F_X(x) - F_X(a)$
    \item $\mathbb{P}(X > a) = 1 - F_X(a)$
    \item $\mathbb{P}(a\leq X\leq b) = F_X(b) - \lim_{x\to a^-}F_X(x)$
    \item $\mathbb{P}(a< X) = \lim_{x\to a^-}F_X(x)$
    \item $\mathbb{P}(X\geq a) = 1 - \lim_{x\to a^-}F_X(x)$
\end{enumerate}
\begin{proof}
Si dimostra solo la 2. Le altre o sono conseguenze o seguono lo stesso schema dimostrativo.
Ricordiamo che dati $C,D\in\mathscr{F}$ tali che $C^c\cap D=\emptyset$ si ha la relazione:
\begin{center}
    $\mathbb{P}(C\setminus D)=\mathbb{P}(C)-\mathbb{P}(D)$
\end{center}
Dunque ponendo:
\begin{center}
    $C=\{\omega\in\Omega$ $|$ $X(\omega)\leq a\}$ e $D=\{\omega\in\Omega$ $|$ $X(\omega)\leq a-\frac{1}{n}\}$ 
\end{center}
Si ottiene così:
\begin{center}
    $\mathbb{P}(X=a)=\lim_{n\to+\infty}\mathbb{P}(C\setminus D)=\lim_{n\to+\infty}\mathbb{P}(C)-\mathbb{P}(D)=$
    \vspace{5px}
    \newline
    $=F_X(a)-\lim_{n\to+\infty}\mathbb{P}(X\leq a-\frac{1}{n})=F_X(a)-\lim_{x\to a^-}F_X(x)$
\end{center}
\end{proof}
\end{proposition}

\vspace{10px}

\newpage
\section{Variabili aleatorie discrete}

\begin{definition}
Una variabile aleatoria $X$ si dice \textbf{discreta} se prende valori in uno spazio $S\subseteq\mathbb{R}$ \textit{numerabile}. $S$ viene detto \textbf{supporto} di $X$.
\end{definition}

\begin{definition}
Data una V.A. $X$ si definisce la funzione:
\begin{center}
    $f:\mathbb{R}\longrightarrow\mathbb{R}$ t.c. $f(x)=\mathbb{P}(X=x)$  $\forall x\in\mathbb{R}$
\end{center}
È detta \textbf{densità discreta}(\textbf{PMF}).
\end{definition}

\vspace{10px}

\begin{observation}
Si osserva subito che:
\begin{itemize}
    \item La densità discreta è nulla su $\mathbb{R}\setminus S$ ed è sempre positiva su $S$
    \item Considerato $S=\bigcup\limits_{i=1}^{+\infty}x_i$ si ottiene:
    \begin{itemize}
        \item $\sum\limits_{i=1}^{+\infty}f(x_i)=1$
        \item $F_X(\hat{x})=\mathbb{P}_X(\bigcup\limits_{i:x_i\leq\hat{x}}\{\omega\in\Omega$ $|$ $X(\omega)=x_i\})=\sum\limits_{i:x_i\leq\hat{x}}f(x_i)$
    \end{itemize}
\end{itemize}
\end{observation}

\vspace{10px}

Verranno ora proposti alcuni importanti esempi di V.A. discrete:

\vspace{10px}

\subsection{Esempi notevoli di v.a. discrete}

\begin{itemize}
    \item \textbf{V.A. di Bernulli} denotata con $X\sim Be(p)$:
    \vspace{5px}
    \newline
    $\Omega=\{Successo(s),Fallimento(f)\}$ con $\mathbb{P}(s)=p$ e $\mathbb{P}(f)=1-p=q$
    \vspace{10px}
    \newline
    \item \textbf{V.A. Binomiale} denotata con $X\sim Bin(n,p)$:
    \vspace{5px}
    \newline
    $\Omega=\{\omega=(\omega_1,\omega_2,...,\omega_n) , \omega_i\in\{s,f\}\}$ di cardinalità finita con
    \newline
    $\mathscr{F}=\mathbf{P}(\Omega)$, $\mathbb{P}(s)=p$ e variabile aleatoria:
    \begin{center}
        $X:\Omega\longrightarrow\mathbb{R}$ t.c. $X(\omega)=$
        "Numero di s nelle n prove"
        $=\sum\limits_{i=1}^n\mathbbm{1}_{\{s\}}(\omega_i)$
    \end{center}
    Dunque il supporto risulta essere: $S=\{1,2,...,n\}$.
    

    
    Si valuta la probabilità del singolo evento elementare con k successi su n prove:
    \begin{center}
        $\mathbb{P}(\{\omega\})=p^k(1-p)^{n-k}$
    \end{center}
    Ottenendo:
    \begin{center}
        $\mathbb{P}(X=k)=\binom{n}{k}p^k(1-p)^{n-k}$
    \end{center}
    \vspace{10px}
    \item \textbf{V.A. Geometrica} denotata $X\sim Ge(p)$:
    \vspace{5px}
    \newline
    $\Omega=\{\omega=(\omega_1,\omega_2,\omega_3...) , \omega_i\in\{s,f\}\}$ di cardinalità infinita con
    \newline
    $\mathscr{F}=\mathbf{P}(\Omega)$, $\mathbb{P}(s)=p$ e variabile aleatoria:
    \begin{center}
        $X:\Omega\longrightarrow\mathbb{R}$ t.c. $X(\omega)=$
        "Numero di prove necessarie affinché esca s"
    \end{center}
    Dunque il supporto risulta essere: $S=\mathbb{N}^*$
    
   Inoltre:
   \begin{center}
       $\mathbb{P}(X=k)=p(1-p)^{k-1}$
   \end{center}
   È sufficiente considerare il primo esito s senza proseguire, infatti se si proseguisse, la probabilità dell'unione sarebbe la somma delle probabilità dei due eventi successivi, che darebbe il medesimo risultato.
   
    \vspace{10px}
    \item \textbf{V.A. Binomiale negativa} denotata $X\sim NBin(p)$:
    \vspace{5px}
    \newline
    $\Omega=\{\omega=(\omega_1,\omega_2,\omega_3...) , \omega_i\in\{s,f\}\}$ di cardinalità infinita con
    \newline
    $\mathscr{F}=\mathbf{P}(\Omega)$, $\mathbb{P}(s)=p$ e variabile aleatoria:
    \begin{center}
        $X:\Omega\longrightarrow\mathbb{R}$ t.c. $X(\omega)=$
        "Numero di prove necessarie affinché esca r volte s"
    \end{center}
    Dunque il supporto risulta essere: $S=\{r,r+1,r+2,...\}$
    
   Inoltre:
   \begin{center}
       $\mathbb{P}(X=k)=\binom{k-1}{r-1}p^k(1-p)^{k-r}$
   \end{center}
   Si ha $\binom{k-1}{r-1}$ poiché l'ultima posizione è già decisa e occupata da s.
   
   \vspace{10px}
   \item \textbf{V.A. ipergeometrica}:
   \vspace{5px}
   \newline
   \noindent
   Si considera l'esperimento probabilistico: Vi sono $N$ individui e una proprietà $\chi$ tali che $S\leq N$ individui hanno $\chi$ e $N-S$ non hanno $\chi$. Si estraggono $n$ individui senza ripetizione. Si chiede quale sia la probabilità che $k$ individui abbiano $\chi$.
   \vspace{5px}
   \newline
   \noindent
   Si considera la V.A. $X$ che conta il numero di individui aventi $\chi$ tra gli $n$ estratti e si osserva:
   \begin{itemize}
       \item Il supporto è: $S=\{0,1,2,...,n\}$
       \item $\mathbb{P}(X=k)=${\Large$\frac{\binom{S}{k}\binom{N-S}{n-k}}{\binom{N}{n}}$}
       Ottenuto ragionando sull'estrazione in blocco.
       \item $\mathbb{P}(X=k)=\binom{n}{k}\frac{S(S-1)...(S-k+1)(N-S)(N-S-1)...(N-S-(n-k)+1)}{N(N-1)...(N-n+1)}$ Ottenuto ragionando sulle estrazioni successive.
   \end{itemize}
   È evidente che le due formule siano uguali.
   
   \vspace{10px}
   \item \textbf{V.A. di Poisson} denotata con $X\sim Po(\lambda)$: 
   \vspace{5px}
   \newline
   \noindent
   La V.A. $X$ di Poisson di parametro $\lambda\in\mathbb{R}_{+}$ ha densità discreta:
   \begin{center}
       $\mathbb{P}(X=k)=$ {\large$\frac{\lambda^k}{k!}e^{-\lambda}$}
   \end{center}
   Si osserva subito che: $\sum\limits_{i=0}^{+\infty}\mathbb{P}(X=k)=\sum\limits_{i=0}^{+\infty}$ {\large$\frac{\lambda^k}{k!}e^{-\lambda}$} $=e^{\lambda}e^{-\lambda}=1$
\end{itemize}

\vspace{15px}

Si osserva una proprietà della V.A. Geometrica, la \textbf{mancanza di} 
\newline
\textbf{memoria}.

\begin{proposition}
Fissato $m>0$ , $X\sim Ge(p)$ allora: 
\begin{center}
$\mathbb{P}(X=k+m)$ $|$ $X>k)=\mathbb{P}(X=m)$    
\end{center}

\vspace{10px}

\begin{proof}
$\mathbb{P}(X=k+m)$ $|$ $X>k)=${\large$\frac{\mathbb{P}(X=k+m\cap X>k)}{\mathbb{P}(X>k)}=\frac{\mathbb{P}(X=k+m)}{\mathbb{P}(X>k)}=$
\vspace{10px}
\newline
$=\frac{p(1-p)^{k+m-1}}{\sum\limits_{j=k+1}^{+\infty}p(1-p)^{j-1}}=\frac{p(1-p)^{k+m-1}}{\sum\limits_{j=1}^{+\infty}p(1-p)^{j-1}-\sum\limits_{j=1}^{k}p(1-p)^{j-1}}=\frac{p(1-p)^{k+m-1}}{1-p\frac{1-(1-p)^k}{1-(1-p)}}=\frac{p(1-p)^{k+m-1}}{(1-p)^k}=$}
\vspace{10px}
\newline
$=p(1-p)^{m-1}=\mathbb{P}(X=m)$
\end{proof}
\end{proposition}

\vspace{10px}

Una correlazione tra V.A. binomiale e di Poisson si può trovare nella seguente proposizione.

\begin{proposition}
Sia $(X)_{n\in\mathbb{N}}$ una successione di V.A. binomiali t.c. 
\newline
$X\sim Bin(n,\frac{\lambda}{n})$ \hspace{4px} $\forall n\in\mathbb{N}^*$ allora:
\begin{center}
    $\lim_{n\to+\infty}\mathbb{P}(X_n = k)=\frac{\lambda^k}{k!}e^{-\lambda}$
\end{center}
\end{proposition}

\vspace{20px}

\subsection{Variabili aleatorie (discrete) bidimensionali}

Si può estendere il concetto di variabile aleatoria, sino a qui legata ad una sola dimensione, a più dimesioni. 

\vspace{15px}

\begin{proposition}
Sia $X$ una V.A. e $g:\mathbb{R}\longrightarrow\mathbb{R}$ una funzione \textit{borel-misurabile}, allora $Y=g\circ X$ è una V.A.
\begin{proof}
Sia $E\in\mathcal{B}(\mathbb{R})$:
\begin{center}
    $Y^{-1}(E)=\{\omega\in\Omega$ $|$ $Y(\omega)\in E\}=\{\omega\in\Omega$ $|$ $g(X(\omega))\in E\}=$
    \newline
    $\{\omega\in\Omega$ $|$ $X(\omega)\in g^{-1}(E)\}$
\end{center}
$g^{-1}(E)\in\mathcal{B}(\mathbb{R})$ poichè g borel-misurabile e dunque $Y^{-1}(E)\in\mathscr{F}$ per def. di V.A.
\end{proof}
\end{proposition}
\vspace{10px}
\begin{definition}
Date $X,Y$ V.A. si definisce:
\begin{itemize}
    \item $(X,Y)$ come \textbf{vettore aleatorio}
    \item \begin{center}
    $F_{(X,Y)}:\mathbb{R}^2\longrightarrow\mathbb{R}$ t.c. $F_{(X,Y)}(x,y)=\mathbb{P}(X\leq x, Y\leq y)$
    \hspace{4px}
    $\forall (x,y)\in\mathbb{R}^2$
    \end{center}
    come \textbf{funzione di distribuzione congiunta} (di probabilità).
    \item Le funzioni $F_x,F_y$ come \textbf{distribuzioni marginali} (di probabilità)
\end{itemize}
\end{definition}

\begin{definition}
Le componenti del vettore aleatorio $(X,Y)$ sono dette \newline \textbf{indipendenti} se:
\begin{center}
    $\mathbb{P}(X\leq x, Y\leq y)=\mathbb{P}(X\leq x)\mathbb{P}(Y\leq y)$
\end{center}
O equivalentemente denotati: $A_x=\{\omega\in\Omega$ $|$ $X(\omega)\leq x\}$ e $B_y=\{\omega\in\Omega$ $|$ $Y(\omega)\leq y\}$
\begin{center}
    $\mathbb{P}(A_x\cap B_y)=\mathbb{P}(A_x)\mathbb{P}(B_y)$
\end{center}
\end{definition}

\vspace{10px}

\begin{observation}
Si può notare che:
    \[\lim_{x\to+\infty}F_{(X,Y)}(x,y)=\lim_{x\to+\infty}\mathbb{P}(X\leq x, Y\leq y)=\]
    \[=\mathbb{P}\Big(\lim_{x\to+\infty}\big(\{\omega\in\Omega | X(\omega)\leq x\}\cap\{\omega\in\Omega | Y(\omega)\leq y\}\big)\Big)=\]
    \[=\mathbb{P}\big(\{\omega\in\Omega | Y(\omega)\leq y\}\big)=F_Y(y)\]
Si ottiene dunque: $\lim_{x\to+\infty}F_{(X,Y)}(x,y)=F_Y(y)$ e $\lim_{y\to+\infty}F_{(X,Y)}(x,y)=F_X(x)$
\end{observation}


\begin{observation}
Osservando che:
\begin{center}
    $F_{(X,Y)}(x,y)=\sum\limits_{i:x_i\leq x}\sum\limits_{j:y_j\leq y}\mathbb{P}(X=x_i, Y=y_j)$ \hspace{6px} dove $(x_i,y_j)\in S$
\end{center}
Inoltre:
\begin{enumerate}
    \item $F_X(x)=\lim_{y\to+\infty}F_{(X,Y)}(x,y)=$
    \vspace{3px}
    \newline
    $\lim_{y\to+\infty}\sum\limits_{i:x_i\leq x}\sum\limits_{j:y_j\leq y}\mathbb{P}(X=x_i, Y=y_j)=\sum\limits_{i:x_i\leq x}\sum\limits_j\mathbb{P}(X=x_i, Y=y_j)$
    \item $F_X(x)=\sum\limits_{i:x_i\leq x}\mathbb{P}(X=x_i)$
\end{enumerate}
Dunque mettendo assieme 1. e 2. si ottiene:
\begin{center}
    $\mathbb{P}(X=x_i)=\sum\limits_j\mathbb{P}(X=x_i,Y=y_j)$
\end{center}
\end{observation}

Altra proprietà degna di nota è la seguente.

\begin{proposition}
Siano $U,V$ due V.A. discrete con funzione di probabilità $\mathbb{P}(U=u,V=v)=p(u,v)$, allora posta $Z=U+V$ si ha: 
\begin{center}
    $\mathbb{P}(Z=z)=\sum\limits_{u}p(u,z-u)$
\end{center}
\begin{proof}
$\mathbb{P}(Z=z)=\mathbb{P}(U+V=z)=\sum\sum\limits_{(u,v) t.c. u+v=z}\mathbb{P}(U=u,V=v)=\sum\sum\limits_{(u,v) t.c. v=u-z}\mathbb{P}(U=u,V=v)=\sum\limits_{u}\mathbb{P}(U=u,V=z-u)$
\end{proof}
\end{proposition}
\newpage
\section{Valore atteso} \label{Val_att_discr}

\begin{definition}
Sia $X$ una variabile aleatoria (discreta) quasi certamente positiva \big($\mathbb{P}(X\geq0)=1$\big), si dice che $X$ è \textbf{integrabile} se:
\[\sum\limits_{i=0}^{+\infty}x_i\mathbb{P}(X=x_i) < +\infty\]
Se poi $X$ è integrabile allora di definisce \textbf{valore atteso} come:
    \[\mathbb{E}X=\sum\limits_{i=0}^{+\infty}x_i\mathbb{P}(X=x_i)\]
\end{definition}

\vspace{10px}

Per estendere la definizione a una V.A. generica mi servo di: \newline
$X^{+}=max(X,0)$ e $X^{-}=min(X,0)$. Entrabe V.A. a valori positivi.

\vspace{10px}

\begin{definition}
Sia $X$ una V.A. discreta a valori reali allora:
    \[\mathbb{E}X=\mathbb{E}X^+-\mathbb{E}X^-\]
\end{definition}
\vspace{5px}
Si osserva immediatamente che: $\mathbb{E}|X|=\sum\limits_{i=0}^{+\infty}|x_i|\mathbb{P}(X=x_i)$.

Iniziali e fondamentali caratterizzazioni del valore atteso sono date dai seguenti teoremi.

\begin{theorem}
Sia $Z$ una V.A. discreta del tipo $$Z=f(X_1,...,X_n)$$ dove $f:\mathbb{R}^n\longrightarrow\mathbb{R}$ è borel-misurabile e $(X_1,...,X_n)$ è vettore aleatorio, allora:
\begin{center}
    \begin{enumerate}
        \item Se $\sum\limits_{i=0}^{+\infty}|f(x_1^i,...,x_n^i)|\mathbb{P}(X_1=x_1^i,...,X_n=x_n^i)<+\infty$ dove $S=\{(x_1^i,...,x_n^i)\}_{i=0}^{+\infty}$ allora $Z$ è integrabile
        \item $\mathbb{E}Z=\sum\limits_{i=0}^{+\infty}f(x_1^i,...,x_n^i)\mathbb{P}(X_1=x_1^i,...,X_n=x_n^i)$
    \end{enumerate}
\end{center}
\begin{proof}
$1.$ Considero $A_j=\{(x_1^i,...,x_n^i)$ $|$ $f(x_1^i,...,x_n^i)=z_j\}=f^{-1}(z_j)$
\vspace{5px}
\newline
Posso riscrivere dunque: \[\{Z=z_j\}=\{\omega\in\Omega | Z(\omega)=z_j\}=\bigcup\limits_{\underline{x}\in A_j}\{X_1=x_1,...,X_n=x_n\}\]
\vspace{5px}
\newline
Ottenendo: $\mathbb{P}(\{Z=z_j\})=\sum\limits_{\underline{x}\in A_j}\mathbb{P}(X_1=x_1,...,X_n=x_n)$
\newline
Dunque:
\[\sum\limits_{j=0}^{+\infty}|z_j|\mathbb{P}(Z=z_j)=
\sum\limits_{j=0}^{+\infty}\sum\limits_{\underline{x}\in A_j}|f(x_1^i,...,x_n^i)| \mathbb{P}(X_1=x_1,...,X_n=x_n)=\]
\[\sum\limits_{i=0}^{+\infty}|f(x_1^i,...,x_n^i)| \mathbb{P}(X_1=x_1^i,...,X_n=x_n^i)\]
\vspace{10px}
\newline
\noindent
$2.$ Equivalente a prima.
\end{proof}
\end{theorem}

\newpage
Il secondo e il terzo teorema definiscono la natura di $\mathbb{E}$.
\begin{theorem}
$\mathbb{E}$ è un operatore lineare.
\begin{proof}
Basta applicare il teorema precedente a $(X,Y)$ con $f(X,Y)=aX+bY$. Ricordarsi di valutare l'integrabilità di $aX+bY$.
\end{proof}
\end{theorem}

\begin{theorem}
$X,Y$ V.A. unidimensionali a valori reali itegrabili: 
\begin{enumerate}
    \item $\mathbb{P}(X\geq Y)=1 \Rightarrow \mathbb{E}X\geq\mathbb{E}Y$
    \item $|\mathbb{E}X|\leq\mathbb{E}|X|$
\end{enumerate}
\begin{proof}
\begin{enumerate}
    \item Pongo $Z=X-Y$. Osservo che: \[\mathbb{P}(X\geq Y) = 1 \Leftrightarrow \mathbb{P}(Z\geq 0) = 1\]
    Dunque il supporto di $Z$ sarà contenuto nei reali \textit{positivi}. Dunque $\mathbb{E}Z\geq 0$ implica $\mathbb{E}X\geq\mathbb{E}Y$
    \item Per monotonia e $-|X|\geq X\geq|X|$ si ottiene: $-\mathbb{E}|X|\geq \mathbb{E}X\geq\mathbb{E}|X|$
\end{enumerate}

\vspace{5px}
\noindent
\end{proof}
\end{theorem}

Si riportano di seguito 3 importanti proprietà del valore atteso, la cui dimostrazione è banale e lasciata al lettore ( ;) ).

\begin{proposition}
Siano $X,Y$ v.a. discrete a valori reali:
\begin{enumerate}
    \item Se $X$ è limitata allora è integrabile
    \item $|X|\leq|Y|$ q.c. allora: $Y$ integrabile $\Rightarrow$ $X$ integrabile
    \item Se $X,Y$ integrabili e \textit{indipendenti} allora $XY$ è integrabile e $\mathbb{E}XY=\mathbb{E}X\mathbb{E}Y$ 
\end{enumerate}
\end{proposition}

\subsection{Momento, Varianza e Covarianza}

\begin{definition}
Sia $X$ una v.a. a valori reali: 
\begin{itemize}
    \item $X$ ammette \textbf{momento di ordine} $k\in\mathbb{N}^*$ se $X^k$ è integrabie
    \item $X$ ammette \textbf{momento centrato di ordine} $k\in\mathbb{N}^*$ se $(X-\mathbb{E}X)^k$ è integrabie
\end{itemize}
Si denotano:
\begin{itemize}
    \item Momento di $X$ di ordine $k$: $\mathbb{E}X^k$
    \item Momento centrato di $X$ di ordine $k$: $\mathbb{E}(X-\mathbb{E}X)^k$
\end{itemize}
\end{definition}

\vspace{10px}

\begin{definition}
Sia $X$ v.a. a valori reali, si definisce \textbf{varianza} il momento centrato di $X$ di ordine 2:
\begin{center}
    $\mathbb{V}_{AR}X=\mathbb{E}(X-\mathbb{E}X)^2$
\end{center}
\end{definition}
\vspace{5px}
\noindent
Si osserva subito che: $\mathbb{V}_{AR}X=\mathbb{E}X^2-(\mathbb{E}X)^2$

\vspace{10px}

\begin{definition}
Siano $X,Y$ v.a. a valori reali, si definisce \textbf{covarianza}:
\begin{center}
    $\mathbb{C}_{OV}(X,Y)=\mathbb{E}[(X-\mathbb{E}X)(Y-\mathbb{E}Y)]$
\end{center}
\end{definition}

\vspace{15px}

\noindent
Una prima importante considerazione da fare è che:

\begin{proposition}
$X$ v.a. ha momento di ordine $k$ finito allora $X$ ha momento di ordine $r$ \hspace{3px} $\forall r\leq k$ 
\begin{proof}
$|X|^r\leq|X|^k+1 \Rightarrow \mathbb{E}|X|^r\leq\mathbb{E}|X|^k+1$ 
\end{proof}
\end{proposition}

\vspace{10px}
\noindent

Di fondamentale importanza, per la caratterizzazione che fornisce dell'operatore $\mathbb{V}_{AR}$, è il seguente teorema.


\newcommand{\V}{\mathbb{V}_{AR}}
\newcommand{\C}{\mathbb{C}_{OV}}
\newcommand{\E}{\mathbb{E}}

\begin{theorem}
Siano $X,Y$ v.a. unidimensionali a valori reali e sia $a\in\mathbb{R}$ allora:
\begin{enumerate}
    \item $\V(aX)=a^2\V X$
    \item $\V(X+a)=\V X$
    \item $\V(X+Y)=\V X+\V Y+2\C(X,Y)$
\end{enumerate}
\begin{proof}
Dimostriamo nell'ordine, ricordando la linearità di $\E$:
\vspace{10px}
\newline
\noindent
$1.)$ $\V(aX)=\E(aX-\E(aX))^2=\E(a^2(X-\E X)^2)=a^2\V X$
\vspace{12px}
\newline
\noindent
$2.)$ $\V(X+a)=\E(X+a-\E(X+a))^2=\E(X+a-\E(X)-a)^2=\V X$
\vspace{1px}
\newline
\noindent
$3.)$ $\V(X+Y)=\E(X+Y-\E(X+Y))^2=\E((X-\E X)+(Y-\E Y))^2=$
\vspace{5px}
\newline
\noindent
$\E(X-\E X)^2+\E(Y-\E Y)^2+2\E((X-\E X)(Y-\E Y))=$
\vspace{5px}
\newline
\noindent
$\V X+\V Y+2\C(X,Y)$
\end{proof}
\end{theorem}

\vspace{10px}
Inoltre nel caso in cui $X$ e $Y$ siano \textit{indipendenti} risulta $\C(X,Y)=0$ e dunque:
\begin{center}
    $\V(X+Y)=\V X+\V Y$
\end{center}

\vspace{15px}

Si introduce ora un'importante funzione nell'ambito dei momenti:

\begin{definition}
Sia $X$ v.a. unidimensionale a valori reali si definisce la \textbf{funzione generatrice dei momenti} come:
\begin{center}
    $m_X(t):\mathbb{R}\longrightarrow\mathbb{R}$ \hspace{3px} t.c. \hspace{3px} $m_X(t)=\E e^{tX}$
\end{center}
\end{definition}

\vspace{10px}

Si osserva che si può stabilire il momento di X di un certo ordine $k$ tramite la sua funzione generatrice dei momenti:
\begin{center}
    $m_X(t)=\sum\limits_ie^{tx_i}\mathbb{P}(X=x_i)=\sum\limits_i\sum\limits_{k=0}^{+\infty}$ {\large$\frac{(tx_i)^k}{k!}$} 
    $\mathbb{P}(X=x_i)=$
    \newline
    $\sum\limits_i\sum\limits_{k=0}^{+\infty}$ {\large$\frac{(t)^k}{k!}$} 
    $x_i^k\mathbb{P}(X=x_i)=\sum\limits_{k=0}^{+\infty}$ {\large$\frac{(t)^k}{k!}$} $\E X^k$
\end{center}

Dunque per la serie di Taylor: $\E X^k=$ {\Large$\frac{\mathrm d^k}{\mathrm d t^k}$}
$\left( m_X(t) \right)|_{t=0}$

\vspace{15px}

Altra caratterizzazione importante è data dalla seguente proposizione.

\begin{proposition}
$X,Y$ v.a. indipendenti allora: $m_{(X+Y)}(t)=m_X(t)m_Y(t)$
\begin{proof}
$m_{(X+Y)}(t)=\E e^{t(X+Y)}=\E (e^{tX}e^{tY})=\E e^{tX}\E e^{tY}=m_X(t)m_Y(t)$
\end{proof}
\end{proposition}

\vspace{5px}

Un esempio dell'importanza e utilità della funzione generatrice dei momenti è data proprio dalla propietà appena dimostrata. Si consiglia di provare a determinare la distribuzione di probabilità della somma di due Poisson di parametro differente tramite la funzione generatrice dei momenti (trovando dunque la Poisson della somma dei parametri).

\subsection{Disuguaglianze}

Si propongono ora alcune importanti disuguaglianze relative allo studio delle variabili aleatorie discrete.

\vspace{15px}


\begin{proposition}
$X$ v.a. non negativa q.c. per $\alpha>0$:
\begin{equation}
    \mathbb{P}(X\geq\alpha)\leq
    {\large \frac{\E X}{\alpha}}
\end{equation}
\label{Dis1}

\vspace{5px}
\begin{proof}
$\E X=\sum\limits_ix_i\mathbb{P}(X=x_i)=\sum\limits_{i:x_i<\alpha}x_i\mathbb{P}(X=x_i)+\sum\limits_{i:x_i\geq\alpha}x_i\mathbb{P}(X=x_i)\geq\sum\limits_{i:x_i\geq\alpha}x_i\mathbb{P}(X=x_i)\geq\alpha\sum\limits_{i:x_i\geq\alpha}\mathbb{P}(X=x_i)=\alpha\mathbb{P}(X\geq\alpha)$
\end{proof}
\end{proposition}

\vspace{10px}

\begin{proposition}[Disuguaglianza di \textbf{Markov}] \label{Dis_Markov}
$X$ v.a. con $\beta>0$ e $k\in\mathbb{N}^*$:
\begin{center}
    $\mathbb{P}(|X|\geq\beta)\leq$
    {\large $\frac{\E|X|^k}{\beta^k}$}
\end{center}
\vspace{5px}
\begin{proof}
Si applichi \ref{Dis1} a $\mathbb{P}(|X|^k\geq\alpha)$ applicando poi il cambio di parametro $\beta=\alpha^k$.
\end{proof}
\end{proposition}

\vspace{10px}

\begin{proposition}[Disguaglianza di \textbf{Chebyshev}]
$X$ v.a. e $\beta>0$ si ha:
\begin{center}
    $\mathbb{P}(|X-\E X|\geq\beta)=$
    {\large $\frac{\V X}{\beta^2}$}
\end{center}
\vspace{5px}
\begin{proof}
Si applichi Markov a $|X-\E X|$ con $k=2$.
\end{proof}
\end{proposition}


\begin{proposition}[Disguaglianza di \textbf{Jensen}]
Sia $X$ v.a. e $\phi$ una funzione convessa si ha:
\begin{center}
    $\varphi(\E X)\leq\E(\varphi(X))$
\end{center}
\vspace{5px}
\begin{proof}
Si dimostra per induzione che $\varphi(\sum\limits_{i=1}^np_ix_i)\leq\sum\limits_{i=1}^np_i\varphi(x_i)$.
\newline
La base dell'induzione con $n=2$ è la definizione di funzione convessa.
\newline
Supponiamo valga per $n-1$:
\vspace{5px}
\newline
$\varphi(\sum\limits_{i=1}^np_ix_i)=\varphi(p_1x_1+(1-p_1)\sum\limits_{i=2}^n\frac{p_i}{1-p_1}x_i)\leq p_1\varphi(x_1)+(1-p_1)\varphi(\sum\limits_{i=2}^n\frac{p_i}{1-p_1}x_i)$
\vspace{5px}
\newline
Si verifica facilmente che $\sum\limits_{i=2}^n${\large$\frac{p_i}{1-p_1}$}$=1$. Dunque posso applicare l'ipotesi induttiva:
\vspace{5px}
\newline
$\varphi(p_1x_1)+(1-p_1)\varphi(\sum\limits_{i=2}^n\frac{p_i}{1-p_1}x_i)\leq p_1\varphi(x_1)+(1-p_1)\sum\limits_{i=2}^n$ {\large$\frac{p_i}{1-p_1}$} $\varphi(x_i)=\sum\limits_{i=1}^np_i\varphi(x_i)$
\vspace{5px}
\newline
Usando questa propietà e ricordandosi la definizione di valore atteso si dimostra la tesi.
\end{proof}
\end{proposition}

\vspace{10px}

\begin{proposition}[Disguaglianza di \textbf{H\"{o}lder}]
Siano $X,Y$ v.a. e $p,q\in\mathbb{R}$ t.c. $\frac{1}{p}+\frac{1}{q}=1$ si ha:
\begin{center}
    $\E |XY|\leq(\E|X|^p)^{\frac{1}{p}}(\E|Y|^q)^{\frac{1}{q}}$
\end{center}
\vspace{5px}
\begin{proof}
Ci si ricorda che presi $a,b\in\mathbb{R}_{\geq0}$ esistono $s,t\in\mathbb{R}$ tali che $a=e^{\frac{s}{p}}$ e $b=e^{\frac{t}{q}}$, dove $\frac{1}{p}+\frac{1}{q}=1$. Poichè $e$ è una funzione convessa si può applicare Jasen:
\begin{center}
    $e^{\frac{s}{p}+\frac{t}{q}}\leq \frac{1}{p}e^s+\frac{1}{q}e^t$
\end{center}
Dunque si ottiene:
\begin{center}
    $ab\leq$ {\large $\frac{a^p}{p}+\frac{b^q}{q}$}
\end{center}

Ponendo $a=${\large $\frac{X}{(\E X^p)^{\frac{1}{p}}}$} e $b=${\large $\frac{Y}{(\E Y^q)^{\frac{1}{q}}}$}, applicando $\E$ (considerandone la monotonia) e ricordando $\frac{1}{p}+\frac{1}{q}=1$ si ottiene:
\begin{center}
    {\large $\frac{\E XY}{\E^{\frac{1}{p}} X^p\E^{\frac{1}{q}} Y^q}$}$\leq 1$
\end{center}
Da cui la tesi.
\end{proof}
\end{proposition}

\vspace{5px}

\begin{proposition}[Disuguaglianza di \textbf{Schwarz}] \label{CSINEQ}
\'E H\"older con $p=q=2$:
\begin{center}
    $\E |XY|\leq(\E |X|^2)^{\frac{1}{2}}(\E |Y|^2)^{\frac{1}{2}}$
\end{center}
\end{proposition}



\begin{proposition}[Disuguaglianza di \textbf{Lyapunov}]
Sia $X$ v.a. e $0<\alpha\leq\beta$:
\begin{center}
    $(\E |X|^{\alpha})^{\frac{1}{\alpha}}\leq(\E |X|^{\beta})^{\frac{1}{\beta}}$
\end{center}
\vspace{5px}
\begin{proof}
Poniamo $p=\frac{\beta}{\alpha},q=\frac{\beta}{\beta-\alpha}$ e $Y:=1$ e $X:=|X|^{\alpha}$ v.a. dunque si applica H\"older.
\end{proof}
\end{proposition}

\newpage

\section{Legge debole dei grandi numeri}

Grazie alle precendenti disuguaglianze possiamo enunciare e dimostrare il seguente teorema.

\vspace{5px}
\newcommand{\xn}{\overline{X}_n}

\begin{theorem}[Bernulli o legge debole dei grandi numeri per v.a. i.i.d.]
Sia $(X_n)_{n\geq 1}$ una successione di v.a. i.i.d (indipendenti identicamente distribuite) e sia $\mu=\E X_n$ e $\sigma^2=\V X_n$ \hspace{2px} $\forall n$. 
\newline 
\noindent
Denominata $\overline{X}_n=\frac{1}{n}\sum\limits_{i=1}^nX_i$ \textbf{media empirica} si ha:
    \[\lim_{n\to+\infty}\mathbb{P}(|\overline{X}_n-\mu|>\eta)=0 \hspace{2px} \forall\eta>0\]
\vspace{5px}
\begin{proof}
Si osserva inizialmente, ricordandosi il fatto che le v.a. sono indenticamente distribuite, che:
\begin{itemize}
    \item $\E\overline{X}_n=\mu$
    \item $\V\overline{X}_n=$ {\large$\frac{\sigma^2}{n}$}
\end{itemize}
Applico Chebyshev a $\xn$:
\begin{center}
    $\mathbb{P}(|\xn-\mu|\geq\eta)\leq$ {\large$\frac{\sigma^2}{n\eta^2}$}
\end{center}
Passando al limite si ottiene la tesi.
\end{proof}
\end{theorem}

\vspace{10px}

\'E utile una considerazione in merito al teorema appena dimostrato. Esso, sotto ristrette condizioni che verranno ammorbidite in seguito (legge forte dei grandi numeri), formalizza e dimostra vera l'intuizione che ci porta a pensare che lanciando un grande numero di volte una moneta la probabilità che uscirà T è circa $\frac{1}{2}$. 

Per vedere meglio l'esempio descritto sopra si può applicare il teorema ad una Bernulli di probabilità $\frac{1}{2}$. 
\newpage
\section{Variabili aleatori assolutamente continue}

\begin{definition}
Una $X$ v.a. unidimensionale a valori in $\mathbb{R}$ si dice \textbf{assolutamente continua} se esiste una funzione $f:\mathbb{R}\longrightarrow\mathbb{R}$ tale che $f$ sia integrabile e:
\[\mathbb{P}(X\leq x)=F_X(x)=\int_{-\infty}^{x} f(t) \,dt \]
$f$ è detta \textbf{funzione di densità di probabilità}.
\end{definition}

\vspace{10px}

Si possono subito fare alcune osservazioni sulla natura delle v.a. ass. cont.

\begin{observation}
\noindent
\begin{itemize}
    \item $\mathbb{P}(a<X\leq b)=\mathbb{P}(X\leq b)-\mathbb{P}(X\leq a)=\int_a^b f(t) \, dt$
    \item $\mathbb{P}(X\in B)=\int_B f(t) \, dt$
    \item $F_X(x)$ \textit{non} è unica: ad esempio basta prendere $g(x)=f(x)$ $\forall x\neq x_0$ per ottenere un integrale uguale.
    \item $F_X(x)$ è continua da destra: $F_X(x_0)-\lim_{x\to x_0^-}F_X(x)=\mathbb{P}(X=x_0)=\int_{x_0}^{x_0} f(t) \, dt=0$
\end{itemize}
\end{observation}

\vspace{15px}

Due importanti tipi di v.a. ass. cont sono i seguenti:
\begin{enumerate}
    \item $X\sim Unif([a,b])$: 
\begin{center}
        $F_X(x) = \left\{
        \begin{array}{ll}
            0 & \quad x < a \\
            \frac{x-a}{b-a} & \quad x \in [a,b) \\
            1 & \quad x \geq b
        \end{array}\right$
\end{center}
Ottenendo dunque:
\begin{center}
    $f(x) = \left\{
        \begin{array}{ll}
            0 & \quad x \notin (a,b) \\
            \frac{1}{b-a} & \quad x \in (a,b) 
        \end{array}
        \right$
\end{center}
Si osserva inoltre che $S=(a,b)$

        \item $X\sim Exp(\lambda)$:
\begin{center}
    $f(x) = \left\{
        \begin{array}{ll}
            \lambda e^{-\lambda x} & \quad x > 0 \\
             0 & \quad x \leq 0 
        \end{array}
        \right$
\end{center}
Ottenendo dunque:
\begin{center}
    $F_X(x) = \left\{
        \begin{array}{ll}
            1- e^{-\lambda x} & \quad x > 0 \\
             0 & \quad x \leq 0 
        \end{array}
        \right$
\end{center}
Dove evidentemente $S=\mathbb{R}_+$
\end{enumerate}

\vspace{5px}

Riguardo al secondo esempio ($X\sim Exp(\lambda)$) si osserva una propietà equivalente alla perdita di memoria nel caso della v.a. discreta geometrica.

\begin{proposition}
Sia $X\sim Exp(\lambda)$ e $s,t\in\mathbb{R}_+$ allora vale: \[\mathbb{P}(X>s+t|X>t)=\mathbb{P}(X>s)\]
\begin{proof}
$\mathbb{P}(X>t+s|X>t)=$ {\Large $\frac{\mathbb{P}(X>t+s,X>t)}{\mathbb{P}(X>t)}=\frac{\mathbb{P}(X>t+s)}{\mathbb{P}(X>t)}=\frac{1-\mathbb{P}(X\leq s+t)}{1-\mathbb{P}(X\leq t)}$}
\vspace{5px}
\newline
{\Large$\frac{1-(1-e^{-\lambda(t+s)})}{1-(1-e^{-\lambda t})}=\frac{e^{-\lambda(t+s)}}{e^{-\lambda t}}$}
{\large $=e^{-\lambda s}$} $=\mathbb{P}(X>s)$
\end{proof}
\end{proposition}

\subsection{Vettori aleatori assolutamente continue}

Come fatto per le v.a. discrete si osserva un'importante proprietà che lega funzioni borel-misurabili e v.a. continue.

\begin{proposition}
Sia $X$ una v.a. ass. cont. e $\varphi:\mathbb{R}\longrightarrow\mathbb{R}$ continua e borel-misurabile allora: $Y:=\varphi(X)$ è una v.a. ass. cont. e inoltre:
\[F_Y(y)=\int_A f_X(t) \,dt\] dove $A=\{t\in\mathbb{R}$ $|$ $t\in\varphi^{-1}((-\infty,y])\}$
\end{proposition}


\begin{definition}
$\textbf{X}=(X_1,...,X_n)$, $n\in\mathbb{N}^*$ è una \textbf{vettore} aleatorio ass. cont. se la sua funzione di distribuzione (congiunta) di probabilità ammette densità:
\[F_\textbf{X}(\overline{x})=\mathbb{P}(X_1\leq x_1,...,X_n\leq x_n)=\int_{-\infty}^{x_1}...\int_{-\infty}^{x_n} f_\textbf{X}(t_1,...,t_n) \,dt_1...dt_n\]
\end{definition}

\vspace{10px}

Sulla falsa riga dei vettori aleatori discreti si enunciano alcune propietà basilari:
\begin{itemize}
    \item $\mathbb{P}(\textbf{X}\in B)=\int_B f_{\textbf{X}}(\overline{x}) \,d\overline{x}$
    \item Marginalizzazione: $f_X(x)=\int_{\mathbb{R}} f_{(X,Y)} \,dy$
    \item Se X,Y indipendenti: $\mathbb{P}(X\leq x,Y\leq y)=\int_{-\infty}^x f_X(x) \,dx\int_{-\infty}^y f_Y(y) \,dy$
\end{itemize}

\vspace{10px}

Analogamente al caso discreto si enuncia e dimostra la seguente propietà.

\begin{proposition}
$(X,Y)$ vettore aleatorio assolutamente continuo con densità $f_{(X,Y)}(x,y)$. La v.a. $Z:=X+Y$ ha densità: \[ f_Z(z) = \int_{-\infty}^{+\infty} f_{(X,Y)}(x,z-x) \, dx \] 
\vspace{5px}
\begin{proof}
\[\mathbb{P}(Z\leq z)=\mathbb{P}(X+Y\leq z)=\int\int_A f(u,v) \, dudv\] 
\vspace{5px}
\newline
dove $A=\{(u,v)\in\mathbb{R}$ $|$ $u+v \leq z\}$. Usando quindi il cambio di coordinate $v=t-u$ , $u=u$ si ottiene:
\[\int\int_A f(u,v) \, dudv = \int_{-\infty}^z\int_{-\infty}^{+\infty} f(u,t-u) \,dudt = \int_{-\infty}^{z}\bigg(\int_{-\infty}^{+\infty} f(u,t-u) \,du\bigg)dt\]
Confrontando l'espressione ottenuta con la definizione di $\mathbb{P}(Z\leq z)$ si ottiene la tesi.
\end{proof}
\end{proposition}


In merito al cambio di variabili di una v.a. ass. cont. si osserva la seguente propietà.

\begin{proposition}
Sia $\underline{X}$ un v.a. ass. cont. con componenti a valori reali e $g:\mathbb{R}^n\longrightarrow\mathbb{R}^n$ una funzione continua e borel-misurabile. 
\newline
Ponendo $\underline{Y}:=g(\underline{X})$ si ottiene:
\[f_{\underline{Y}}(\overline{y})=f_{\underline{X}}(g^{-1}(\overline{y}))|DetJ_{g^{-1}}(\overline{y})| \]
\vspace{5px}
\begin{proof}
Sia $A\in\mathcal{B}(\mathbb{R})$
\[\mathbb{P}(\underline{Y}\in A) = \int_A f_{\underline{Y}}(\overline{y}) \,d\overline{y} = \int_{g^{-1}(A)} f_{\underline{X}}(\overline{x}) \,d\overline{x}\]
Ricordando i teoremi dell'analisi sul cambio di variabile negli integrali multipli e applicando il cambio di variabile $\overline{x}=g^{-1}(\overline{y})$:
\[\int_{g^{-1}(A)} f_{\underline{X}}(\overline{x}) \,d\overline{x} = \int_{A} f_{\underline{X}}(g^{-1}(\overline{y}))|DetJ_{g^{-1}}(\overline{y})| \,d\overline{y}\]
Ottenendo la tesi.
\end{proof}
\end{proposition}

\vspace{10px}

Un importante uso della propietà appena illustrata si trova, ad esempio, nel seguente esercizio:
\vspace{5px}
\newline
Siano $X,Y$ due v.a. tali che $X\sim Y\sim Exp(\lambda)$ si trovi $f_V(v)$ e $f_U(u)$ dove $U=X$,$V=$ {\large $\frac{X}{Y}$}.
\vspace{5px}
\newline
Per risolvere tale esercizio si può considerare $(U,V)$ come cambio di variabile di $(X,Y)$, trovando $f_{(U,V)}$, per poi marginalizzare.

\newpage

\subsection{Momenti nelle v.a. assolutamente continue}

Rimandando all'analoga trattazione del capitolo \ref{Val_att_discr}.
\begin{definition}
$X$ v.a. ass. cont. con densità $f_X$ è integrabile se: \[\int_{-\infty}^{+\infty} |x|f_X(x) \,dx < +\infty\]
\end{definition}

\vspace{10px}

\begin{definition}
Se $X$ v.a. ass. cont. a valori reali positivi integrabile si ha:
\[\E X^+=\int_{-\infty}^{+\infty} xf_X(x) \,dx\]
\vspace{5px}
Si definisce: \[\E X = \E X^+ - \E X^-\]
\end{definition}


\vspace{15px}

\begin{observation}
Se si ha un $\underline{X}$ vettore aleatorio ass. cont. integrabile e $\varphi:\mathbb{R}^n\longrightarrow\mathbb{R}$ una funzione continua e borel-integrabile allora posta $Z=\varphi(\underline{X})$ si ha: 
\begin{center}
    $Z$ è integrabile $\Longleftrightarrow \int_{-\infty}^{+\infty} |\varphi(\overline{x})|f_{\underline{X}}(\overline{x}) \,d\overline{x} < +\infty $
\end{center}
E si ha: \[\E Z = \int_{-\infty}^{+\infty} \varphi(\overline{x})f_{\underline{X}}(\overline{x}) \,d\overline{x}\]
\end{observation}

\newpage

\subsection{Esempi notevoli di v.a. assolutamente continue}

\vspace{5px}

Si presentano ora, come fatto per il caso discreto, una serie di variabili aleatorie assolutamente continue di significativa importanza.

\vspace{5px}


\begin{itemize}
    \item \textbf{V.A. Normale} (Gaussiana) \textbf{di parametri 0 e 1} denotata con
    \newline
    $Y\sim N(0,1)$:\[ f_Y(y)={ \frac{e^{-\frac{y^2}{2}}}{\sqrt{2\pi}}} \]
Richiamando i risultati ottenuti nel corso di Analisi 2 si osserva che $f_Y$ integrata su tutto $\mathbb{R}$ da come risultato 1. 

Nel caso tali risultati siano difficili da richiamare si tratta di risolvere il quadrato dell'integrale tramite un passaggio in coordinate polari.

Inoltre notando la positività di $f_Y$ si verifica la natura di densità di probabilità.
    
    \vspace{10px}
    \item \textbf{V.A. Normale di parametri $\mu$ e $\sigma^2$} denotata con $X\sim N(\mu,\sigma^2)$:
    \vspace{5px}
    \newline
    Presa $Y\sim N(0,1)$ si pone $X:=\sigma X + \mu$ ottenendo:
    \begin{center}
    $F_X(x)=F_Y\big(${\large $\frac{x-\mu}{\sigma}$}\big)    
    \end{center}
    Dunque: 
    \begin{center}
        $f_X(x) =$ {\Large $\frac{1}{\sigma\sqrt{2\pi}}$ {\Large $e^{-\frac{(x-\mu)^2}{2\sigma^2}}$ } } 
    \end{center}
    Inoltre tramite le propietà del valore atteso e della varianza, calcolando $\E Y$ e $\V Y$, si ottiene:
    \begin{itemize}
        \item $\E X = \mu$
        \item $\V X = \sigma^2$
    \end{itemize}
    
    
    \newcommand{\intt}{\int_{-\infty}^{+\infty}}
    
    Calcoliamo ora la funzione generatrice dei momenti di una v.a. normale:
    
    \[\phi_X(t) = e^{(\mu+\frac{\sigma^2t^2}{2})}\]
    
    Poichè:
    
    \[\phi_X(t)=\E e^{tX}=\int_{-\infty}^{+\infty}e^{tx} \frac{{\Huge e^ \frac{-(x-\mu)^2}{2\sigma^2}}}{\sigma \sqrt{2\pi}}  \,dx =\]
    \[ \frac{1}{\sigma\sqrt{2\pi}}\intt e^{\frac{-(x^2+\mu^2-2x\mu-2tx\sigma^2+2\mu\sigma^2t+\sigma^4t^2)}{2\sigma^2}} \cdot e^{\frac{(2\mu\sigma^2t+\sigma^4t^2)}{2\sigma^2}} \,dx =\]
    \[\frac{e^{(\mu+\frac{\sigma^2t^2}{2})}}{\sqrt{2\pi}} \intt e^{\frac{-\big(y-\frac{(\mu-\sigma^2t)}{\sigma}\big)^2}{2}} \,dy = \frac{e^{(\mu+\frac{\sigma^2t^2}{2})}}{\sqrt{2\pi}} \intt e^{\frac{-z^2}{2}} \,dz = e^{(\mu+\frac{\sigma^2t^2}{2})} \] 
    
    \vspace{10px}
    \newpage
    Grazie alla caratterizzazione data dalla f.g.m. possiamo valutare la v.a. $Z=X+Y$ dove $X\sim N(\mu_1,\sigma_1^2), Y\sim N(\mu_2, \sigma_2^2)$:
    
    \[\E e^{t(X+Y)}=\E e^{tX}\E e^{tY}=e^{(\mu_1+\frac{\sigma_1^2t^2}{2})}e^{(\mu_2+\frac{\sigma_2^2t^2}{2})}=e^{(\mu_1+\mu_2)+\frac{(\sigma_1^2+\sigma_2^2)t^2}{2}}\]
    
    Ottendendo che $Z\sim N(\mu_1+\mu_1,\sigma_1^2+\sigma_2^2)$
    
    \vspace{10px}
    \item \textbf{V.A. Gamma ($\Gamma$)} denotata con $X\sim Ga(\alpha,\lambda)=\Gamma(\alpha,\lambda)$:
    \vspace{5px}
    \newline
    Si richiama:
    \begin{align*}
        \Gamma\colon \mathbb{R_+} & \longrightarrow \mathbb{R_+} \\
        \alpha& \longmapsto \int_0^{+\infty} x^{\alpha-1}e^{-x} \,dx
    \end{align*}
    
    Dunque di definisce, ponendo $\alpha>0$,$\lambda>0$: $x>0$ \[f_X(x) = \frac{\lambda^{\alpha}}{\Gamma(\alpha)}x^{\alpha-1}e^{-\lambda x} \hspace{3px}\]

     Si nota subito che: $\Gamma(1,\lambda)\sim Exp(\lambda)$
     
     \vspace{5px}
     
     Si dice che $X\sim \Gamma(\frac{n}{2},\frac{1}{2})$ si distribuisce come un \textbf{chi-quadro} ($\chi^2$) con $n$ gradi di libertà.
     
     \vspace{5px}
     
     Osserviamo ora due imprtanti proprietà della v.a. gamma e chi-quadro.
     
     \begin{proposition}
     $Z\sim N(0,1)$ allora $Y=Z^2\sim \chi_1^2$
     \begin{proof}
     Ricordiamo che per $X\sim\chi_1^2$ si ha: $f_X(x)=\frac{x^{-\frac{1}{2}}e^{-\frac{1}{2}x}}{2^{\frac{1}{2}}\Gamma(\frac{1}{2})}=\frac{1}{\sqrt{2\pi x}}e^{-\frac{1}{2}x}$
     Dunque: \[\mathbb{P}(Y\leq y)=\mathbb{P}(Z^2\leq y)=\mathbb{P}(-\sqrt{y}\leq Z\leq\sqrt{y})=F_Z(\sqrt{y})-F_Z(-\sqrt{y})\]
     Ottenendo: \[f_Y(y)=\frac{d}{dy}\Big(F_Z(\sqrt{y})-F_Z(-\sqrt{y})\Big)=\frac{1}{2\sqrt{y}}\big(2f_Z(\sqrt{y})\big)=\frac{1}{\sqrt{y}}\frac{1}{\sqrt{2\pi}}e^{-\frac{y}{2}}\]
     \end{proof}
     \end{proposition}
     
     \begin{observation}
     Si ricordano:
     \begin{itemize}
         \item $X\sim Bin(n,\theta)\longrightarrow m_X(t)=[\theta e^t+(1-\theta)]^n$
         \item $X\sim \Gamma(\theta_1,\theta_2)\longrightarrow m_X(t)=\Big(${\large $\frac{\theta_2}{\theta_2-t}$}$\Big)^{\theta_1}$
         \item $X\sim N(\theta_1,\theta_2)\longrightarrow m_X(t)=exp(\theta_1t+\theta_2\frac{t^2}{2})$
         \item $X\sim \Gamma(\frac{p}{2},\frac{1}{2})\longrightarrow m_X(t)=(1-2t)^{\frac{p}{2}}$
     \end{itemize}
     \end{observation}
     
     \vspace{15px}
     
     \begin{proposition}
     Siano $Y_1,...,Y_n$ v.a. ind. t.c. $Y_i\sim\chi_{p_i}^2$ per $i=1,...,n$ allora si ha che $\sum\limits_{i=1}^nY_i\sim\chi_p^2$ dove $p=\sum\limits_{i=1}^np_i$
    \begin{proof}
    Basta osservare la f.g.m chiamando $Y=\sum\limits_{i=1}^nY_i$: \[m_{Y_i}(t)=(1-2t)^{\frac{p_i}{2}} \Rightarrow m_Y(t)=\prod_{i=1}^nm_{Y_i}(t)=\prod_{i=1}^n(1-2t)^{\frac{p_i}{2}}=(1-2t)^p\]
    \end{proof}
     \end{proposition}
     
     \vspace{10px}
     
     \newcommand{\Aa}{\alpha}
    \newcommand{\Bb}{\beta} 
     -Studiamo ora i momenti della v.a. $X\sim\Gamma(\alpha,\lambda)$:
     
     \[\E X^{\beta} = \int_0^{+\infty} x^{\beta}\frac{\lambda^{\Aa}}{\Gamma(\Aa)}x^{\Aa-1}e^{-\lambda x} \,dx = \frac{\lambda^{\Aa}}{\Gamma(\Aa)}\int_0^{+\infty}\Big(\frac{y}{\lambda}\Big)^{\Aa+\Bb-1}e^{-y}\frac{1}{y} \,dy\] 
     \[=\frac{\lambda^{-\Bb}}{\Gamma(\Aa)}\int_0^{+\infty}y^{\Aa+\Bb-1}e^{-y} \,dy = \frac{\Gamma(\Aa+\Bb)}{\lambda^{\Bb}\Gamma(\Aa)}\]
     Dunque:
     \begin{itemize}
         \item $\E X=${\large $\frac{\Aa}{\lambda}$}
         \item $\E X^2=${\large $\frac{\Aa(\Aa+1)}{\lambda^2}$}
     \end{itemize}
     Ottenendo: \[\V X=\frac{\Aa}{\lambda^2}\]
     \vspace{10px}
     \item \textbf{V.A. T di student} con $p$ gradi di libertà, denotata con $T\sim t_p$:
     \newline
     Ha funzione di densità: \[f_T(t)=\frac{\Gamma(\frac{p+1}{2})}{\Gamma(\frac{p}{2})}\frac{1}{\sqrt{p\pi}}\frac{1}{(1+\frac{t^2}{p})^{\frac{p+1}{2}}}\]
     Prima fondamentale caratteristica è che: $T$ $non$ ha f.g.m.

     $T\sim t_1$ è detta di Cauchy.
     
     Un importante teorema che lega distribuzione normale e  chi-quadro alla student è il seguente.
     \begin{theorem}
     Siano $U\sim N(0,1)$ e $V\sim\chi_p^2$ allora $T=\frac{U}{\sqrt{\frac{V}{p}}}\sim t_p$
     \begin{proof}
     Per dimostrale l'asserto si applichi il cambiamento di variabili \newline $\varphi(u,v)=(\frac{u}{\sqrt{\frac{v}{p}}},v)$. 
     
     Quindi si razionalizzi rispetto alla prima componente e si risolva l'integrale ricoduncendo l'integrando alla funzione di densità di una \newline $\Gamma((p+1)/2,\frac{1}{2}(1+t^2/p))$.
     \end{proof}
     \end{theorem}
     
     \vspace{10px}
     \item \textbf{V.A. F di Fisher} con gradi di libertà $(p,q)$, denotata con $W\sim F_{p,q}$:
     
     Ha una funzione di densità: \[f_F(f)=\frac{\Gamma(\frac{p+q}{2})}{\Gamma(\frac{p}{2})\Gamma(\frac{q}{2})}\bigg(\frac{p}{q}\bigg)^{\frac{p}{2}}\frac{\omega^{\frac{p}{2}-1}}{\Big[1+\big(\frac{p}{q}\big)\omega\Big]^{\frac{p+q}{2}}}\]
     
     Prima importante caratterizzazione si ha con: $T\sim t_p \Rightarrow T^2\sim F_{1,q}$
     
     Un'altra caratterizzazione la si ha tramite il seguente teorema.
     \begin{theorem}
     Siano $U\sim\chi_p^2$ e $V\sim\chi_q^2$ allora $W=\frac{\frac{U}{p}}{\frac{V}{q}}\sim F_{p,q}$
     \end{theorem}
     \begin{proof}
     Per dimostrale l'asserto si applichi il cambiamento di variabili \newline
     $\varphi(u,v)=(\frac{u/p}{v/q},v/q)$
     Quindi si razionalizzi rispetto alla prima componente e si risolva l'integrale tramite $t=x(pw+q)$ e ricoduncendo l'integrando alla funzione di densità di un $\chi_{\frac{p+q}{2}}$.
     \end{proof}
\end{itemize}


\newpage
\section{Funzione caratteristica}

\begin{definition}
Date $Z_1,Z_2$ v.a. a vaori reali allora $Z=Z_1+iZ_2$ è una v.a. definita sullo stesso spazio probabilistico detta a \textbf{valori complessi}.
\end{definition}

\vspace{5px}

\begin{observation}
Se $Z_1$ e $Z_2$ sono integrabili lo è anche $Z$ e vale: \[\E Z=\E Z_1 +i\E Z_2\]
\end{observation}

\vspace{10px}

Possiamo dunque definire la funzione caratteristica.
\begin{definition}
Sia $\textbf{X}$ una v.a. a valori reali in $\mathbb{R}^d$ si dice \textbf{funzione caratteristica di \textbf{X}} la: 
\begin{align*}
    \varphi\colon \mathbb{R}^d & \longrightarrow \mathbb{C} \\
    \theta&\mapsto \E e^{i\langle\theta,\textbf{X}\rangle} = \E cos(\langle\theta,\textbf{X}\rangle) + i\E sin(\langle\theta,\textbf{X}\rangle)
\end{align*}
\end{definition}

\vspace{5px}

Una serie di basilari osservazioni:
\begin{itemize}
    \item $\varphi(0)=1$
    \item $X$ discreta: \[\varphi(\theta)=\sum_k e^{i(\theta\cdot k)}\mathbb{P}(X=k)\]
    \item $X$ ass. cont: \[\varphi(\theta)=\int_{\mathbb{R}^d}e^{i(\theta\cdot \textbf{x})}f_\textbf{X}(\textbf{x}) \,d\textbf{x}\]
\end{itemize}

\vspace{15px}

Di fondamentale importanza il seguente.

\begin{theorem}
Siano $X,Y$ v.a:
\begin{center}
    $X,Y$ identicamente distribuite $\Longleftrightarrow$ $X,Y$ hanno la stessa funzione caratteristica
\end{center}
\end{theorem}

\vspace{10px}
\newcommand{\Tt}{\theta}

Enunciamo (la dimotrazione è lasciata come esercizio) brevemente due propietà:
\begin{enumerate}
    \item $X,Y$ ind. $\Rightarrow$ $\varphi_{X+Y}(\theta)=\varphi_X(\Tt)\varphi_Y(\Tt)$
    \item $Y=AX+b$ allora: $\varphi_Y(\Tt)=e^{i\Tt\cdot b}\varphi_{X}(A^t\Tt)$
\end{enumerate}

\vspace{5px}

In analogia con la f.g.m si osserva la seguente propietà che lega i momenti e la funzione caratteristica:
\[\E X^k = \frac{1}{i^k}\frac{d^k}{d\Tt^k}\varphi_X(\Tt)|_{\Tt=0}\]
Per ricarvale questo risultato si procede analogamente alla f.g.m., cioè usando l'espansione di Taylor in un intorno di $0$.
\newpage
\section{Convergenze e Limite centrale}

Si illustrano in questo conclusivo capitolo sui fondamenti della teoria della probabilità alcune basilari tipologie di convergenze, concludendo con l'enunciato e la dimostrazione del teorema del limite centrale.

\vspace{10px}

\subsection{Convergenze}

Si presentano ora i tre fondamentali tipi di convergenze di una successione di variabili aleatorie, utili nella nostra trattazione a presentare il teorema del limite centrale.

\vspace{15px}
\newline
\noindent
Sia $(X_n)_{n\in\mathbb{N}}$ una successione di v.a. a valori reali, si dice che:

\begin{definition}
$(X_n)_{n\in\mathbb{N}}$ \textbf{converge in probabilità} alla v.a. X se:
\begin{center}
$\forall \eta > 0$ \hspace{4px} $\mathbb{P}(X_n-X>\eta)\xrightarrow{n\to+\infty}0$    
\end{center}
O equivalentemente:
\begin{center}
$\forall \epsilon > 0$ \hspace{4px} $\mathbb{P}(X_n-X<\epsilon)\xrightarrow{n\to+\infty}1$
\end{center}
Si denota con: $X_n\xrightarrow{\mathbb{P}}X$.
\end{definition}

\vspace{5px}

\begin{definition}
$(X_n)_{n\in\mathbb{N}}$ \textbf{converge quasi-certamente} alla v.a. X se:
\begin{center}
    $\mathbb{P}(\{\omega\in\Omega$ $|$ $X_n(\omega)\xrightarrow{n\to+\infty}X(\omega)\}) = 1$
\end{center}
Si denota con: $X_n\xrightarrow{q.c.}X$.
\end{definition}

\vspace{5px}

\begin{definition}
$(X_n)_{n\in\mathbb{N}}$ \textbf{converge in distribuzione} alla v.a. X se:
\[\lim_{x \to +\infty} F_{X_n}(x) = F_X(x)\]
Per ogni $x$ punto di continuità di $F_X$ 
%\vspace{2px}
\newline
\noindent
Si denota con: $X_n\xrightarrow{d}X$
\end{definition}


Tra le convergenze proposte vale una gerarchia ben precisa, stabilita dal seguente teorema:

\vspace{5px}

\begin{theorem}
La convergenza quasi-certa implica la convergenze in probabilità che implica la convergenze in distribuzione.
\end{theorem}



Un ulteriore importante teorema, ultimo elemento fondamentale per il teorema centrale del capitolo, è il teorema di Lèvy.

\begin{theorem}[Lèvy]
Sia $(X_n)_{n\in\mathbb{N}}$ una successione di v.a. con funzione caratteristica $\varphi_n(\Tt)$ per ogni $n$ e sia X una v.a. con funzione caratteristica $\varphi(\Tt)$, vale:
\begin{center}
$X_n\xrightarrow{d}X \Longleftrightarrow \varphi_n(\Tt)\xrightarrow{n\to+\infty}\varphi(\Tt)$ \hspace{4px} $\forall\Tt\in\mathbb{R}$
\end{center}
\end{theorem}

\vspace{15px}

\subsection{Teorema del limite centrale}

Possiamo finalmente enunciare e dimostrare il teorema del limite centrale, che ci servirà come ponte tra la teoria della probabilità e la statistica. Il seguente teorema lega in una relazione due elementi che a priori sembrerebbero slegati. Seppur sotto ipotesi restrittive, la condizione di IID, il risultato è inaspettato e permette di estendere notevolmente lo studio delle variabili aleatorie. Il solo fatto che io stia facendo una parabola introduttiva al teorema dovrebbe far suonare qualche campanello, vista l'austerità degli appunti.

\vspace{10px}
\noindent
Si ricorda la notazione: \[\overline{X}_n=\frac{\sum\limits_{i=0}^n X_i}{n}\]

\begin{theorem}
Sia $(X_n)_{n\in\mathbb{N}}$ una successione di v.a. IID con $\mu=\E X_1$ e $\sigma^2=\V X_1$, allora:
\begin{center}
$S_n=${\large $\frac{\overline{X}_n-\mu}{\frac{\sigma}{\sqrt{n}}}$} $\xrightarrow{d} Z\sim N(0,1)$
\end{center}

\begin{proof}
Valutiamo:
\begin{itemize}
    \item $\E \overline{X}_n = \frac{\sum\limits_{i=0}^n\E X_i}{n}=\mu$
    \item $\V \overline{X}_n = \frac{\sum\limits_{i=0}^n\V X_i}{n^2} =$ {\Large$ \frac{\sigma^2}{n}$}
\end{itemize}
Normalizzando $X_n$ ottengo: $Y_n=${\large $\frac{X_n-\mu}{\sigma}$}
\begin{itemize}
    \item $\E Y_n = 0$
    \item $\V Y_n = \frac{1}{\sigma^2}\V X_n = 1$
\end{itemize}
Pongo ora: $S_n=${\large $\frac{1}{\sqrt{n}\sigma}$} $\sum\limits_{i=0}^n(X_i-\mu)=${\large $\frac{1}{\sqrt{n}}$}$\sum\limits_{i=0}^nY_n$
\newline
\noindent

Valuto quindi la funzione caratteristica di $S_n$:
\vspace{10px}
\newline
\[
\varphi_{S_n}(t)=\E{\large e^{itS_n}}=\E{\large e^{\big(i\frac{t}{\sqrt{n}}\sum_{i=0}^nY_n\big)}}=\varphi_{(\sum_{i=0}^nY_n)}\bigg(\frac{t}{\sqrt{n}}\bigg)=\]
\begin{equation}
\label{eq_3}
\prod\limits_{i=0}^n\varphi_{Y_i}\bigg(\frac{t}{\sqrt{n}}\bigg)=\bigg[\varphi_{Y_1}\Big(\frac{t}{\sqrt{n}}\Big)\bigg]^n    
\end{equation}

\vspace{10px}

Dove nella penultima uguaglianza si usa l'indipendenza e nell'ultima uguaglianza l'identica distribuzione (IID).
\newline
Per poter successivamente valutare il polinomio di Taylor centrato in $0$ si determinano ora le prime due derivate di $\varphi_{Y_1}(\frac{t}{\sqrt{n}})$ in $t=0$ tramite il teorema che lega lafunzione caratteristica e i momenti di una v.a:
\begin{itemize}
    \item $\varphi_{Y_1}(0) = 1$
    \item $\dot{\varphi}_{Y_1}(0) = i\E Y_1 = 0$
    \item $\ddot{\varphi}_{Y_1}(0) = i^2\E Y_1^2 = -\V Y_1 = -1$
\end{itemize}
Ottenendo per $x\rightarrow 0$:
\[\varphi_{Y_1}(x) = \varphi_{Y_1}(0) + \dot{\varphi}_{Y_1}(0)x + \ddot{\varphi}_{Y_1}(0)\frac{x^2}{2} + o(x^2) = 1 - \frac{x^2}{2} + o(x^2)\]
Nonchè per $n\rightarrow +\infty $:
\[\varphi_{Y_1}\bigg(\frac{t}{\sqrt{n}}\bigg) = 1 - \frac{t^2}{2n} + o\bigg(\frac{1}{n}\bigg) \]
Si osserva che l'o-piccolo è in funzione solo di $\frac{1}{n}$ poichè t si intende come constante.
\newline
Ora possiamo sfruttare \ref{eq_3} ottenendo:
\begin{center}
    $\varphi_{S_n}(t)=[\varphi_{Y_1}(\frac{t}{\sqrt{n}})]^n=${\large$e^{n\cdot ln(\varphi_{Y_1}(\frac{t}{\sqrt{n}}))}\sim e^{n\cdot ln(1 - \frac{t^2}{2n} + o(\frac{1}{n}))}\sim e^{n(-\frac{t^2}{2n}+o(\frac{1}{n}))}\sim e^{\frac{-t^2}{2}}$}
\end{center}
Dove $e^{\frac{-t^2}{2}}$ è la funzione caratteristica di una Normale standard, e dunque usando Lèny otteniamo la tesi.
\end{proof}
\end{theorem}
\part{Statistica}
\chapter{Definizioni}
\begin{definition}
Sia $n\in\mathbb{N}^*$ e $\underline{X}=(X_1,...,X_n)$ un vettore aleatorio di componenti IID con densità (discreta o continua) $f(\cdot;\theta)$. Si dice $\underline{X}$ \textbf{campione casuale} di taglia $n$ estratto da una popolazione (genitrice) $X$ con densità $f(\cdot;\theta)$.
\end{definition}

\vspace{10px}

Si può considerare $\underline{X}$ come un esperimento aleatorio ripetuto n volte.

\vspace{10px}

\begin{definition}
Sia $X$ una popolazione di un campione casuale caraterrizzata da una densità $f(\cdot,\theta)$. Sia $\mathcal{X}$ il supporto di $X$ ($\mathcal{X}\in\mathbb{R}^n$) allora la famiglia di densità di probabilità $\{f(\cdot;\theta) : \theta\in\Theta\}$, parametrizzata da $\theta$ è detta \textbf{modello statistico}.
\newline
Il supporto di $\underline{X}$ è detto \textbf{spazio dei campioni}.
\end{definition} 

\vspace{5px}

\begin{observation}
Si nota che:\[f_{(X_1,...,X_n)}(x_1,...,x_n)=\prod\limits_{i=1}^nf(x_i;\theta)\]
\end{observation}

\vspace{15px}

\begin{definition}
Sia $(X_1,...,X_n)$ un campione casuale e sia $G:\mathbb{R}^n\longrightarrow\mathbb{R}^m$ con $m\geq 1$ borel-misurabile $non$ dipendente da $\theta$ è detta \textbf{statistica}.
\end{definition}

\vspace{10px}

\begin{definition}
Sia $(X_1,...,X_n)$ un campione casuale estratto da una popolazione $X$ con densità $f(\cdot;\theta)$. Il \textbf{momento campionario} è definito come:
\[M_k=\frac{1}{n}\sum_{i=1}^nX_i^k\]
Con $k\in\mathbb{N}^*$
\end{definition}

\vspace{15px}

Denotando ora $\mu_k=\E X^k$ possiamo enunciare e dimostrate il seguente teorema.

\begin{theorem}
Sia $(X_1,...,X_n)$ il campione casuale di taglia $n$ estratto da una popolazione $X\sim f(\cdot,\theta)$ allora:
\begin{enumerate}
    \item $\E M_k=\mu_k$
    \item $\V M_k=\frac{1}{n}(\mu_{2k}-\mu_k^2)$
\end{enumerate}
\begin{proof}
\begin{enumerate}
    \item \[\E M_k=\E\bigg(\frac{1}{n}\sum_{i=1}^nX_i^k\bigg)=\frac{1}{n}\sum_{i=0}^n\E X_i^k=\mu_k\]
    Dove nell'ultimo passaggio si usa la identica distribuzione.
    \item \[\V M_k=\V \bigg(\frac{1}{n}\sum_{i=1}^nX_i^k\bigg)= \frac{1}{n^2}\sum_{i=1}^n\V X_i^k=\]
    \[=\frac{1}{n}\sum_{i=0}^n(\E X_i^{2k}-\E^2 X_i^k)=\frac{1}{n}(\mu_{2k}-\mu_k^2)\]
    Dove nel secondo passaggio si usa l'idipendenza delle v.a.
\end{enumerate}
\end{proof}
\end{theorem}

\vspace{10px}

Si vuole fornire un campione casuale di un nuovo "concetto" di momento.

\begin{definition}
Sia $(X_1,...,X_n)$ un campione casuale di taglia $n$ estratto da una popolazione $X$ con densità $f(\cdot,\theta)$, si definisce \textbf{varianza campionaria} la quantità: \[S^2=\frac{1}{n-1}\sum_{i=1}^n(X_i-\overline{X}_n)^2\]
\end{definition}
\vspace{10px}

La varianza campionaria viene messa in relazione alla varianza del campione tramite il seguente teorema, dove $\sigma^2=\V X$.

\newcommand{\sliin}{\sum\limits_{i=1}^n}
\newcommand{\slin}{\sum\limits_{i=0}^n}
\newcommand{\sliinf}{\sum\limits_{i=1}^{+\infty}}
\newcommand{\slinf}{\sum\limits_{i=0}^{+\infty}}
\begin{theorem} \label{teor_20}
Sia $(X_1,...,X_n)$ un campione casuale di taglia $n$ estratto da una popolazione $X\sim f(\cdot,\theta)$ allora:
\begin{enumerate}
    \item $\E S^2=\sigma^2$ 
    \item $\V S^2=\frac{1}{n}(\mu_4-\frac{n-3}{n-1}\sigma^4)$
\end{enumerate}

\begin{proof}
Si dimostra solo la 1.
\newline
\begin{enumerate}
    \item Si osserva:
    \begin{center}
        $\sliin(X_i-\overline{X}_n)^2=\sliin(X_i-\mu+\mu-\overline{X}_n)^2=$
        \newline
        $\sliin\Big( (X_i-\mu)^2+(\overline{X}_n-\mu)^2-2(X_i-\mu)(\overline{X}_n-\mu))=\sliin(X_i-\mu)^2+n(\overline{X}_n-\mu)^2-2(\overline{X}_n-\mu)\sliin(X_i-\mu)=$
        \newline
        $\sliin(X_i-\mu)^2+n(\overline{X}_n-\mu)^2-2(\overline{X}_n-\mu)n(\overline{X}_n-\mu)=\sliin(X_i-\mu)^2-n(\overline{X}_n-\mu)^2$
    \end{center}
    Si ottiene così:
    \begin{center}
        $\E S^2=\E \bigg(\frac{1}{n-1}\sliin(X_i-\overline{X}_n)\bigg)=\frac{1}{n-1}\E\bigg(\sliin(X_i-\overline{X}_n)\bigg)=$
        \vspace{5px}
        \newline
        $\frac{1}{n-1}\E\bigg(\sliin(X_i-\mu)^2-n(\overline{X}_n-\mu)^2\bigg)=\frac{1}{n-1}\bigg(\sliin\E(X_i-\mu)^2-n\E(\overline{X}_n-\mu)^2\bigg)=\frac{1}{n-1}(n\sigma^2-\sigma^2)=\sigma^2$
    \end{center}
    \vspace{5px}
    Dove: $\E (\overline{X}_n-\mu)^2=\V \overline{X}_n=\frac{1}{n}\V X=\frac{1}{n}\sigma^2$
\end{enumerate}
\end{proof}
\end{theorem}


\newpage
\chapter{Metodi inferenziali}
\section{Stima puntuale}

Come accenna il Casella-Berger nell'intrudizione del capitolo relativo, il concetto di stimatore è inizialmente molto vago tanto da fornire come definizione iniziale:

\begin{definition}
Uno \textbf{stimatore} di un campione casuale è una qualsiasi $statistica$ di esso. 
\end{definition}

Si precisa un'imprtante differenza tra $stima$ e $stimatore$: 
\begin{center}
    Una \textbf{stima} è la realizzazione di uno \textbf{stimatore}.
\end{center}

\vspace{15px}

Determinare degli stimatori non è sempre pratica facile, anche se in alcuni casi la sola intuizione può risultare sufficiente per compiere un lavoro dignitoso. Si presentano di seguito due metodologie classiche e con risultati differenti per lo studio di stimatori.

\subsubsection{Metodi per determinare stimatori}
\begin{itemize}
    \item Il più antico, classico, dal risultato assicurato ma ricco di limitazioni è il \textbf{metodo dei momenti}:

Sia $(X_1,...,X_n)$ un c.c. estratto da una popazione con densità $f(\cdot;\theta_1,...,\theta_k)$. Il medoto consiste nel uguagliare i momenti della popolazione:\[\mu_r=\E X^r\] con i momenti campionari: \[M_r=\frac{1}{n}\sum_{i=0}^nX_i^r\] per $r\in\mathbb{N}^*$. 

Risolvendo il sistema risultante, ottenendo dunque un vettore $(\hat{\theta}_1,...,\hat{\theta}_k)$ di $k$ elementi in funzione di $(\theta_1,...,\theta_k)$, si stabiliscono gli stimatori.

\vspace{10px}

Si suggerisce come esercizio di determinare, tramite il metodo dei momenti, gli stimatori di:
\begin{itemize}
    \item un c.c. di taglia $n$ estratto da una popolazione normale $X\sim N(\mu,\sigma^2)$
    \item un c.c. di taglia $n$ estratto da una popolazione uniforme in  \newline $(\mu-\sigma\sqrt{3},\mu+\sigma\sqrt{3})$
\end{itemize} 
Ottenendo il medesimo risultato di $\hat{\mu}=\overline{X}_n$ e $\hat{\sigma}=(M_2-\overline{X}_n^2)^{\frac{1}{2}}$.
    \item Il secondo metodo che analizziamo è il \textbf{metodo della massima verosimiglianza}:
    
Per introdurre questo metodo ci serviamo della funzione di verosimiglianza.
\begin{definition}
Sia $\underline{X}$ un c.c. di taglia $n$ e funzione di densità congiunta $f_{\underline{X}}(\overline{x},\theta)$, la \textbf{funzione di verosimiglianza} è la funzione di densità congiunta avente come parametri $\theta$: \[L(\theta)=\prod_{i=1}^nf_X(x_i;\theta)\] 
\end{definition}
Dunque si fissa $(x_1,...,x_n)$ e si valuta la variazione di $f_{\underline{X}}(x_1,...,x_n;\theta)$.

Osserviamo inoltre, per comprendere lo stimatore di massima verosimiglianza, che $L(\theta_1)\geq L(\theta_2)$ implica che, rispetto ad uno specifico $(x_1,...,x_n)$, in $\theta_1$ si ha più $verosimiglianza$ (probabilità) che in $\theta_2$.

\begin{definition}
Sia $L(\theta)$ una funzione di verosimiglianza relativa al c.c. $(x_1,...,x_n)$. Sia $\hat{\theta}=\hat{\theta}(x_1,...,x_n)\in\Theta$ che $massimizza$  $L(\theta)$, dunque $\hat{\theta}_{ML}=\hat{\theta}(x_1,...,x_n)$ è detto \textbf{stimatore di massima verosimiglianza} per $\theta$.

Con la ovvia estensione nel caso di dimensioni maggiori.
\end{definition}

Come esempio fondamentale si può considerare la seguente situazione e calcolarne lo stimatore con il metodo di massima verosimiglianza:

Ci sono palline nere e palline bianche e ne vengono estratte $n$, si può rappresentare la situazione con $\{Be(p), p\in[0,1]\}$ e derivando la funzione di verosimiglianza per trovare i punti critici.
\end{itemize}


\vspace{15px}

Si introducono ora due importanti caratterizzazioni degli stimatori, che \newline valutano la "qualità" di essi.

\begin{definition}
Sia $T=g(X_1,...,X_n)$ uno stimatore del parametro $\theta$:
\begin{itemize}
    \item $T$ è \textbf{corretto} se: $\E T=\theta$
    \item Si dice \textbf{distorsione} la quantità: $b(T)=\E T-\theta$
    \item Si dice \textbf{errore quadratico medio} la quantità: $MSE(T)=\E (t-\theta)^2$
\end{itemize}
\end{definition}

\vspace{5px}

Si osserva subito che: \[MSE(T)=\E(T-\theta)^2=\E(T-\E T +\E T-\theta)^2=\]
\[\E(T-\E T)^2+2\E\big((T-\E T)(\E T-\theta)\big)+\E(\E T-\theta)^2=\V T +b(T)^2\]

Avendo come obiettivo ottenere una $MSE(T)$ prossimo a zero e sapendo che varianza e distorsione sono $non$ negativi si cercano degli stimatori con queste quantità basse.

Un classico esempio di stimatore privo di distorsione, corretto, è la varianza campionaria (corretta), il cui studio del momento primo è stato affrontato nel teorema \ref{teor_20}.
\newpage
\section{Stimatori corretti uniformemente a varianza minima}

In questo capitolo si vuole chiarire e rendere rigoroso il concetto di efficienza di uno stimatore.

Si evidenzia subito un importante fatto:
\begin{center}
    \textbf{Non} esiste il $miglior$ stimatore
\end{center}
Esistono piuttosto stimatori più adatti alla situazione e raccomandati, in funzioni di parametri variabili. Il motivo di questa affermazione risisede nel fatto che la classe delgi stimatori sia troppo larga. Infatti un comune approccio alla ricerca degli stimatori consiste nel limitare la propria attenzione ai soli stimatori \textbf{corretti}. in questo modo possiamo limitare la nostra attenzione allo studio della varianza degli stimatori, ricordando come obiettivo la minimizzazione dell' MSE.

%\vspace{5px}

Cominciamo con il dare delle definizioni fondamentali.
\begin{definition}
Siano $T_1,T_2$ due stimatori corretti per il parametro $\theta$, si definisce \textbf{efficienza relativa} il rapporto: \[\frac{\V T_2}{\V T_1}\]
\end{definition}

\vspace{5px}

\begin{definition}
Uno stimatore $T$ corretto per il parametro $\theta$ è detto \textbf{efficiente} o \textbf{uniformemente a varianza minima} se: \[\V T\leq \V T'\]
$\forall T'$ stimatore corretto di $\theta$, $\forall\theta\in\Theta$.
\end{definition}


Il teorema di Cramer-Rao ci fornisce di alcune informazioni, in particolare stabilisce un bound inferiore, della varianza di uno stimatore che soddisfa alcune ipotesi.

\begin{theorem}
Dato un campione casuale di taglia $n$ estratto da una popolazione $X$ con densità $f(\cdot;\theta)$. Sia $T=g(X_1,...,X_n)$ uno stimatore corretto di $\theta$, se:
\begin{enumerate}
    \item $\frac{\partial}{\partial\theta}(log(f(x;\theta)))$ esiste $\forall x,\theta$ \label{1}
    \item $\frac{\partial}{\partial\theta}\int f(x;\theta) \,dx = \int \frac{\partial}{\partial\theta} f(x;\theta) \,dx$ \label{2}
    \item $\frac{\partial}{\partial\theta}\big(\int...\int g(x_1,...,x_n)\prod\limits_{i=1}^n f(x;\theta)\,dx_1...dx_n\big)=\int...\int g(x_1,...,x_n)\frac{\partial}{\partial\theta}\prod\limits_{i=1}^n f(x;\theta) \,dx_1...dx_n$ \label{3}
    \item $\E \Big[\Big(\frac{\partial}{\partial\theta}log\big(f(X;\theta)\big)\Big)^2\Big]$ è finito $\forall\theta\in\Theta$ \label{4}
\end{enumerate}

Allora:
    \item \[ \V T \geq \frac{1}{n\E[(\frac{\partial}{\partial\theta}log(f(X;\theta))^2]}\]
Inoltre l'uguaglianza vale sse $\exists\alpha(\theta)$ t.c. $T=\theta+\alpha(\Tt)\sum\limits_{j=1}^n\frac{\partial}{\partial\theta}log(f(X_j;\theta))$
\begin{proof}
Poniamo: \[X=T-\theta\] e \[Y=\frac{\partial}{\partial\theta}log\bigg(\prod_{i=1}^nf(X_i;\theta)\bigg)\] che esiste per \ref{1}.

Usando Cauchy-Swartz \ref{CSINEQ}:
\begin{equation} \label{3.0}
   \E \Big[\big(T-\theta\big)\Big(\frac{\partial}{\partial\theta}log\big(\prod_{i=1}^nf(X_i;\theta)\Big)\Big] \leq \E(T-\theta)^2\E\Big(\frac{\partial}{\partial\theta}log(\prod_{i=1}^nf(X_i;\theta)\Big)^2 
\end{equation}
Valuto ora la parte sinistra della disuguaglianza:
\begin{equation} \label{3.1}
   \E \Big[\big(T-\theta\big)\Big(\frac{\partial}{\partial\theta}log\Big(\prod_{i=1}^nf(X_i;\theta)\Big)\Big]=\E \Big[T\Big(\frac{\partial}{\partial\theta}log\big(\prod_{i=1}^nf(X_i;\theta)\Big)\Big]-\theta\E \Big[\frac{\partial}{\partial\theta}log\big(\prod_{i=1}^nf(X_i;\theta)\Big] 
\end{equation}
Di cui il primo addendo:
\[\E \Big[T\Big(\frac{\partial}{\partial\theta}log\big(\prod_{i=1}^nf(X_i;\theta)\big)\Big)\Big]=\]
\[\int...\int g(x_1,...,x_n)\Big(\frac{\partial}{\partial\theta}log\big(\prod\limits_{i=1}^n f(x_i;\theta)\big)\Big) f(x_1,...,x_n) \,dx_1...dx_n=\]
\[\int...\int g(x_1,...,x_n)\Big(\frac{\partial}{\partial\theta}log\big(\prod\limits_{i=1}^n f(x_i;\theta)\big)\Big) \prod\limits_{i=1}^n f(x_i;\theta) \,dx_1...dx_n=\]
\[\int...\int g(x_1,...,x_n)\frac{1}{\cancel{\prod\limits_{i=1}^n f(x_i;\theta)}}\frac{\partial}{\partial\theta}\big(\prod\limits_{i=1}^n f(x_i;\theta)\big)\cancel{\prod\limits_{i=1}^n f(x_i;\theta)}\,dx_1...dx_n=\]
\[\int...\int g(x_1,...,x_n)\frac{\partial}{\partial\theta}\big(\prod\limits_{i=1}^n f(x_i;\theta)\big)\,dx_1...dx_n\]
Usando \ref{3}:
\[\int...\int g(x_1,...,x_n)\frac{\partial}{\partial\theta}\big(\prod\limits_{i=1}^n f(x_i;\theta)\big)\,dx_1...dx_n=\]
\[\frac{\partial}{\partial\theta}\bigg(\int...\int g(x_1,...,x_n)\big(\prod\limits_{i=1}^n f(x_i;\theta)\big)\,dx_1...dx_n\bigg)=\]
\[\frac{\partial}{\partial\theta} \E T=1\]


Seguendo un ragionamento del tutto analogo, le uguaglianze sono le stesse di prima senza $g(x_1,...,x_n)$, si ottiene che è nullo (derivata di 1). Dunque la \ref{3.1} vale 1.

Ora valutiamo il secondo fattore a destra della disuguaglianza \ref{3.0}:
\[\E\bigg[\Big(\frac{\partial}{\partial\theta}log(\prod_{i=1}^nf(X_i;\theta)\Big)^2\bigg]=\E\bigg[\Big(\sum\limits_{i=1}^n\frac{\partial}{\partial\theta}log(f(X_i;\theta)\Big)^2\bigg]=\]
\[\E\Big(\sum\limits_i\frac{\partial}{\partial\theta}log(f(X_i;\theta)\sum\limits_j\frac{\partial}{\partial\theta}log(f(X_j;\theta)\Big)=\]
\[\E\Big(\sum\limits_i\sum\limits_j\frac{\partial}{\partial\theta}log(f(X_i;\theta)\frac{\partial}{\partial\theta}log(f(X_j;\theta)\Big)=\sum\limits_i\sum\limits_j\E\Big(\frac{\partial}{\partial\theta}log(f(X_i;\theta))\frac{\partial}{\partial\theta}log(f(X_j;\theta))\Big)\]
\newpage
Ora se:
\begin{itemize}
    \item $i\neq j$ si può fattorizzare il prodotto ottenendo due elementi della forma del secondo addendo della \ref{3.1}, quindi ottenendo $0$. Si presti attenzione, si sta usando $f(X_i;\Tt)$, \textbf{non} $f(x_i;\Tt)$.
    \item $i=j$ si ottiene:
    \[\E\Big(\frac{\partial}{\partial\theta}log(f(X_i;\theta))\frac{\partial}{\partial\theta}log(f(X_j;\theta))\Big)=\E\Big(\big(\frac{\partial}{\partial\theta}log(f(X_i;\theta))\big)^2\Big)\]
\end{itemize}
Dunque riprendendo la catena di uguaglianze:
\[\sum\limits_i\sum\limits_j\E\Big(\frac{\partial}{\partial\theta}log(f(X_i;\theta))\frac{\partial}{\partial\theta}log(f(X_j;\theta))\Big)=\sum\limits_{i=1}^n\E\Big(\big(\frac{\partial}{\partial\theta}log(f(X_i;\theta))\big)^2\Big)=\]
\[n\E\Big(\big(\frac{\partial}{\partial\theta}log(f(X;\theta))\big)^2\Big)\]

Dunque mettendo tutto assieme, ricordandosi che $\V T=\E(T-\theta)^2$ poichè $T$ è corretto, si ottiene la tesi.
\end{proof}
\end{theorem}

\vspace{15px}

Conseguenza del teorema è la seguente.
\begin{proposition}
$T$ è uno stimatore corretto per $\theta$ con varianza che raggiunge il limite inferiore di Cramer-Rao, allora $T$ è stimatore di massima verosimiglianza.
\begin{proof}
Basta usare la condizione necessaria sufficiente dell'uguaglianza di Cramer-Rao, ricordarsi che $L(\theta)=\prod_{i=1}^nf(x_i;\theta)$, porre uguale a zero la derivata della \textit{log-verosimiglianza} e trovare così che essa si annulla per $\Tt=T$ e quindi:
\[\hat{\Tt}_{ML}=T\]
\end{proof}
\end{proposition}
\newpage
\section{Proprietà asintotiche e sufficienza di uno stimatore}
\subsection{Proprietà asintotiche}
Com'è naturale che avvenga ci si interroga sulla natura degli stimatori di campioni causuali di taglia tendente all'infinito. Infatti noi abbiamo sempre considerato c.c. di taglia $n\in\mathbb{N}^*$, ma viene naturale immaginarsi come csi comportano in termini di correttezza gli stimatori per c.c. di taglia $n\longrightarrow+\infty$. 

Iniziamo con il fornire due definizioni.

\vspace{5px}

\begin{definition}
Uno stimatore $T_n=g(X_1,...,X_n)$ per il parametro $\Tt$ si dice \textbf{asintoticamente corretto} se:
\[\lim_{n\to+\infty}\E T_n=\Tt\]
\end{definition}

\vspace{5px}

Si osserva subito che la correttezza implica la correttezza asintotica. 

\begin{definition}
Uno stimatore $T_n=g(X_1,...,X_n)$ è detto \textbf{consistente} per il parametro $\Tt$ se $\forall\epsilon>0$:
\[\lim_{n\to+\infty}\mathbb{P}(|T_n-\Tt|<\epsilon)=1\]
Nonchè: $T_n\xrightarrow{\mathbb{P}}\Tt$.
\end{definition}

\vspace{10px}

Un importante teorema che lega la consistenza con la correttezza asintotica è il seguente.

\begin{theorem}
Sia $T_n$ uno stimatore asintoticamente corretto per $\Tt$ e con $\V T_n$ finita $\forall n\in\mathbb{N}^*$:
\begin{center}
    Se $\V T_n\xrightarrow{n\to+\infty}0$ allora $T_n$ è consistente.
\end{center}
\begin{proof}
Ricordando la disuguaglianza di Markov \ref{Dis_Markov}:
\[0\leq \mathbb{P}(|T_n-\Tt|\geq\epsilon)\leq \frac{\E(T_n-\Tt)^2}{\epsilon^2}=\frac{\V T_n + \E^2(T_n-\Tt)}{\epsilon^2}\xrightarrow{n\to+\infty}0\]
Dove l'ultimo limite deriva dall' ipotesi sulla varianza e sulla correttezza asintotica di $T_n$.
\end{proof}
\end{theorem}

\subsection{Sufficienza}

Come detto nell'introduzione sugli stimatori uno stimatore è una qualsiasi statistica. Questa definizione, così generale, ha però delle pecche, o meglio, non tiene conto di quando uno stimatore possa essere considerato sufficientemente valido o meno. Per colmare questo aspetto della teoria sugli stimatori ci viene in soccorso la definizione di \textit{sufficienza} di uno stimatore.

\vspace{5px}

\begin{definition}
Sia $(X_1,...,X_n)$ un c.c estratto da una popolazione con densità $f(\cdot;\Tt)$, una statistica $S=s(X_1,...,X_n)$ e detta \textbf{sufficiente} se la distribuzione condizionata di $(X_1,...,X_n)$ dato ${S=s_0}$ $non$ dipende da $\Tt$, $\forall s_0\in SUPP(S)$.
\end{definition}

\vspace{5px}

Un banale ma esplicativo esempio può essere condotto dal lettore analizzando un campione bernulliano di taglia 3 e valutando la sufficienza della statistica che somma gli elementi e della statistica che somma il terzo con il prodotto dei primi due (osservando che il primo risulta sufficiente mentre il secondo no).

Un importante teorema, del quale non vedremo la dimostrazione, che fornisce una condizione necessaria e sufficiente per la sufficienza di una statistica è il seguente.

\begin{theorem}[Fattorizzazione o Fisher-Neyman]
Sia $(X_1,...,X_n)$ un c.c. con genitrice $X$ di densità $f(x;\Tt)$. La statistica $S=s(X_1,...,X_n)$ è sufficiente se e solo se: \[f_{(X_1,...,X_n)}(x_1,...,x_n)=g(s(x_1,...,x_n);\Tt)h(x_1,...,x_n)\]
Dove $h$ è non negativa e \textit{non} dipendente da $\Tt$ mentre $g$ è non negativa e dipendente da $(x_1,...,x_n)$ \underline{solo} tramite $s(x_1,...,x_n)$.
\end{theorem}
\newpage
\section{Stima intervallare}

Sino ad ora abbiamo considerato solo ed esclusivamente la stima puntuale, dove i valori che vengono stimati sono punti. In questa sezioni ci occupiamo di affrontare la stima intervallare dove, crazy as it sounds, vengono stimati degli intervalli di valori, dunque ci chiediamo quando un parametro rientra in un interavllo.

Cominciamo con il dare la definizione fondamentale.

\begin{definition}
Sia $\overline{X}$ un c.c. di taglia $n$ estratto da una popolazione $X$ con densità $f(x;\Tt)$ e siano $T_1,T_2$ due statistiche tali che $\mathbb{P}(T_1<T_2)=1$. Considerato $\mathbb{P}(T_1<\Tt<T_2)=1-\alpha$, dove $1-\alpha$ non dipende da $\Tt$ e $\alpha\in(0,1)$ si definisce:
\begin{itemize}
    \item $[T_1,T_2]$ \textbf{intervallo di confidenza} di livello $(1-\alpha)$ per $\Tt$
    \item $(1-\alpha)$ \textbf{livello di confidenza}
\end{itemize}
Inoltre l'itervallo con estremi le realizzazioni delle statistiche è anch'esso detto itervallo di confidenza.
\end{definition}

\vspace{5px}
\noindent
Come esempio si consideri la seguente situazione:

Si prenda un c.c. di taglia 4 estratto da una normale di cui si conosce la varianza ($\sigma^2)$. Voglio determinare l'intervallo di confidenza per $\mu$ (il valore attesto) di livello $(1-\alpha)$. Per far ciò calcolo $a,b\in\mathbb{R}$ (i percentili) per cui $\mathbb{P}(a<Z<b)=1-\alpha$ dove $Z=\frac{\overline{X}_n-\mu}{\frac{\sigma}{\sqrt{n}}}\sim N(0,1)$. Dunque riconduco \newline $\mathbb{P}(a<Z<b)=1-\alpha$ a $\mathbb{P}(T_1<\mu<T_2)=1-\alpha$, ottenendo così gli entremi dell' itervallo di confidenza.

\vspace{10px}

\subsection{Metodo della quantità pivotale}

Un importante fattore che caratterizza gli stimatori intervallari è la loro dipendenza o meno dal parametro di cui cercano la stima. lo studio dei metodi pivotali ha proprio come oggetto stimatori \textit{non} dipendenti dal parametro.

\begin{definition}
$Q=Q(X_1,...,X_n;\Tt)$ funzione del c.c. e del paramtero $\Tt$ con funzione di distribuzione \textit{non} dipendente da $\Tt$ è detta \textbf{quantità pivotale}.
\end{definition}

Dunque se $X\sim f(x;\Tt)$ allora $Q(X,\Tt)$ ha la stessa distribuzione per ogni $\Tt$.

Due quantità pivotali abbastanza generali vengono individuate dal prossimo teorema.


\begin{theorem}
Dato un c.c. di taglia $n$ estratto da una popolazione con funzione di densità $f(\cdot;\Tt)$ e funzione di probabilità $F(x;\Tt)$ e continua rispetto a $x$, dunque: 
\begin{center}
    $\prod\limits_{i=1}^nF(X_i;\Tt)$ e $-\sum\limits_{i=1}^nlog(F(X_i;\Tt))$
\end{center}
Sono quantità pivotali.

\begin{proof}
Valuto la la funzione di probabilità della funzione della probabilità, che è ovviamente distribuita come una uniforme su $[0,1]$.

Dunque poniamo $Y=-log(F(X_i;\Tt))$ e ne studiamo la funzione di probabilità:
\[\mathbb{P}(Y\leq y)=1-\mathbb{P}(F(X_i;\Tt)<e^{-y})=\begin{cases} 1-e^{-y}, & 0<e^{-y}\leq1 \\ 0, & 1<e^{-y}
\end{cases}\]
Dunque osserviamo che $Y\sim Exp(1)$. Noi sappiamo inoltre che la somma di esponenziali con parametro $\lambda$ e indipendenti è una Gamma di parametri $(n,\lambda)$. Dunque: \[-\sum\limits_{i=1}^nlog(F(X_i;\Tt))\sim\Gamma(n,1)\]
Confermando così la tesi.

Per quanto riguarda la prima quantità mostriamo che può essere riscritta in funzione della seconda:
\[\mathbb{P}\bigg(\prod_{i=1}^nF(X_i;\Tt)\leq y\bigg)=\mathbb{P}\bigg(log\Big(\prod_{i=1}^nF(X_i;\Tt)\Big)\leq log(y)\bigg)=\]
\[\mathbb{P}\bigg(\sum\limits_{i=1}^nlog(F(X_i;\Tt))\leq log(y)\bigg)=1-\mathbb{P}\bigg(\sum\limits_{i=1}^n-log(F(X_i;\Tt))<-log(y)\bigg)\]
Dunque anche la prima quantità è pivotale.
\end{proof}
\end{theorem}
\newpage
\section{Test delle ipotesi}

Ultimo metodo inferenziale che si presenta, oltre alla stima puntuale e intervallare, è il \textit{test delle ipotesi}. La seguente introduzione sarà priva della classica suddivisione \textit{definizione-commento}, preferendo una trattazione leggermente più discorsiva.

Iniziamo a sottolineare che:
\begin{center}
    \FirstLine{un'ipotesi è fatta su una popolazione.}
\end{center}

Infatti l'obiettivo del metodo è di stabilire, basandosi su un campione osservato, quale di due ipotesi complementari è vera.
\vspace{5px}
Le due ipotesi complementari sono dette:
\begin{itemize}
    \item $H_0$ : $\Tt\in\Theta_0$ ipotesi \textbf{nulla}
    \item $H_1$ : $\Tt\in\Theta_1$ ipotesi \textbf{alternativa}
\end{itemize}
Dove $\Theta_0,\Theta_1$ sono sottoinsiemi disgiunti dello spazio dei parametri ($\Theta$). Inoltre le ipotesi si dicono \textbf{semplici} se $H_0$ : $\Tt = \Tt_0$ e/o $H_1$ : $\Tt = \Tt_1$. Si dicono \textbf{composte} altrimenti.

Dunque un test delle ipotesi, dopo aver raccolto un campione statistico, deve stabilire se valutare $H_0$ come vera e $H_1$ come falsa o viceversa. Così facendo, il test partiziona lo spazio dei campioni in:
\begin{itemize}
    \item $C$ Regione di \textbf{rifiuto}
    \item $C^c$ Regione di \textbf{accettazione}
\end{itemize}

Nel decidere se accettare o meno l'ipotesi nulla si può incorrere in errori, ed è proprio in base alla probabilità di effettuarli che si valutano e comparano i test.

Si distiguono subito due tipologia di errore:
\begin{enumerate}
    \item Se $\Tt\in\Theta_0$ ma il test rifiuta $H_0$ allora si è commesso un errore \textbf{del primo tipo} : $\{\underline{X}\in C ; \Tt\in\Theta_0\}$
    \item Se $\Tt\in\Theta_1$ ma il test accetta $H_0$ allora si è commesso un errore \textbf{del secondo tipo} : $\{\underline{X}\in C^c ; \Tt\in\Theta_1\}$
\end{enumerate}
Poichè gli errori sono eventi possiamo definire le probabilità: $\alpha = \mathbb{P}(\{\underline{X}\in C ; \Tt\in\Theta_0\})$ e $\beta = \mathbb{P}(\{\underline{X}\in C^c ; \Tt\in\Theta_1\})$.
\vspace{5px}
Si osserva subito che $\alpha = 1 - \beta$, dando senso alla definizione di \textbf{potenza del test}:
\[\Pi(\Tt)=\mathbb{P}(\underline{X}\in C ; \Tt\in\Theta) = \left\{
     \begin{array}{lr}
       \alpha &  \Tt \in \Theta_0\\
       1-\beta &  \Tt\in\Theta_1
     \end{array}
   \right.\]
   
Altro parametro che fornisce importanti informazioni su di un test è la sua \textbf{ampiezza}: \[\Tilde{\alpha}=\sup\limits_{\Tt\in\Theta_0}\Pi(\Tt)\]   
Come ultima caratterizzazione di un test ci soffermiamo sul concetto di test \textit{più potente}.
   
\begin{definition}
   Un test $Y$ di ampiezza $\Tilde{\alpha}$ per $H_0$ e $H_1$ si dice \textbf{più potente} se per ogni altro test $Y'$ di ampiezza $\Tilde{\alpha}'\leq\Tilde{\alpha}$ si ha: $\beta'(\Tt)\geq\beta(\Tt)$ per ogni $\Tt\in\Theta_1$
\end{definition}
   
   
Al fine di determinare la regione critica di una test, fissata la sua ampiezza, ci viene in aiuto il seguente teorema.

\begin{theorem}[Lemma di Neymann-Pearson]
Sia $\underline{X}$ un c.c. di taglia $n$ estratto da una genitrice parametrizzata da $\Tt$, e sia $L(\Tt)$ la funzione di verosimiglianza associata. Il test più potente di ampiezza $\alpha$ per verificare $H_0 : \Tt=\Tt_0$ e $H_1 : \Tt=\Tt_1$ è quello avente come regione critica: \[C=\bigg\{(x_1,...,x_n) : \frac{L(\Tt_0)}{L(\Tt_1)}\leq k\bigg\}\]
Dove $k>0$ tale che il test abbia ampiezza $\alpha$.
\begin{proof}
Sia $C$ la regione critica della tesi: $\mathbb{P}(\underline{X}\in C ; H_0 vera)=\alpha$:
\newline 
Poniamo invece $D$ come una regione critica qualunque di ampiezza $\alpha$ e dimostriamo che $C$ ha probabilità dell'errore di seconda specie minore. Valutiamo $\alpha$:
\[\alpha=\int...\int_C L(\Tt_0) \,dx_1...dx_n = \int...\int_D L(\Tt_0) \,dx_1...dx_n  \]
Ricordando $C=(C\cap D)\cup(C\cap D^c)$ l'uguaglianza sopra è equivalente a:
\[\int...\int_{C\cap D^c} L(\Tt_0) \,dx_1...dx_n =\int...\int_{D\cap C^c} L(\Tt_0) \,dx_1...dx_n =  \]
Sfruttando ora l'ipotesi su $C$, e la conseguente ipotesi cu $C^c$:
\[\int...\int_{C\cap D^c} L(\Tt_1) \,dx_1...dx_n \geq \int...\int_{C\cap D^c} \frac{L(\Tt_0)}{k} \,dx_1...dx_n = \]
\[\int...\int_{D\cap C^c} \frac{L(\Tt_0)}{k} \,dx_1...dx_n \geq \int...\int_{D\cap C^c} L(\Tt_1) \,dx_1...dx_n\]
Ottenendo così: \[\int...\int_C L(\Tt_1) \,dx_1...dx_n\geq\int...\int_{D\cap C^c} L(\Tt_1) \,dx_1...dx_n + \int...\int_{C\cap D} L(\Tt_1) \,dx_1...dx_n\]
Dunque: \[\mathbb{P}(\underline{X}\in C ; H_1 vera) \geq \mathbb{P}(\underline{X}\in D ; H_1 vera)\]
Cioè:\[1-\beta_c\geq 1-\beta_d\]
E quindi la tesi.
\end{proof}
\end{theorem}

\newpage

\setcounter{secnumdepth}{-1}
    \myChapter{Conclusioni}


Si concludono qui le lezioni, e dunque i miei appunti, del corso di \textit{Calcolo delle probabilità e statistica} dell'anno accademico 2020/2021. 

Questi sono i primi e si spera non ultimi appunti che scrivo in {\LaTeX}. Sono felice di averli fatti, anche se hanno richiesto un buon numero di ore di lavoro. Sono molto scarni di commenti, esempi e spiegazioni più discorsive. Principalmente perchè non avevo voglia di farle, e in secondo luogo, ma neanche tanto secondo, perchè non le so fare. Tutto ciò che viene spiegato in queste pagine è stato infatti appoggiato dalla lettura dei due libri di riferimento per il corso, il Casella-Berger e il Grimmett-Stirzaker, ed è a loro che rimando per una comprensione più approfondita della materia.

Spero queste note siano state utili a chiunque ne sia in possesso e mi auguro possiate superare l'esame con un bel 29, non 30 che poi magari pensate anche di essere bravi. 

Un affettuoso saluto,

\vspace{5px}
\noindent
Andrea Scalenghe.
\end{document}