\begin{definition}
    Siano $(A,\mathcal{A})$ e $(E,\mathcal{E})$ due spazi misurabili, una funzione $f:\mathcal{A}\longrightarrow\mathcal{E}$ è detta \textbf{misurabile} se:
    \begin{center}
         $B\in\mathcal{E} \Rightarrow f^{-1}(B)\in\mathcal{A}$ \hspace{4px} $\forall B\in\mathcal{E}$ 
    \end{center}
\end{definition}

\vspace{5px}

\begin{definition}
    Sia $(\Omega,\mathscr{F},\mathbb{P})$ uno spazio probabilistico e $(E,\mathcal{E})$ uno spazio misurabile, una funzione $X:\Omega\longrightarrow\mathcal{E}$ \textit{misurabile} è una \textbf{variabile aleatoria}.
\end{definition}

\vspace{10px}

\begin{example}
    Si prende come spazio misurabile $(\mathbb{R},\mathcal{B}(\mathbb{R}))$ (i reali con i suoi boreliani). 
    \begin{itemize}
        \item $\mathbb{P}(X=3)=\mathbb{P}(X^{-1}(\{3\}))$
        \item $\mathbb{P}(X\geq7)=\mathbb{P}(X^{-1}([7,+\infty)))$
        \item $\mathbb{P}(X^{-1}(B))=\mathbb{P}(\{\omega\in\Omega$ $|$ $X(\omega)\in B\})$ \hspace{4px} $\forall B\in\mathcal{B}(\mathbb{R})$
    \end{itemize}
\end{example}

\vspace{10px}

\begin{theorem}
    Sia $(\Omega,\mathscr{F},\mathbb{P})$ uno spazio probabilistico, $(E,\mathcal{E})$ uno spazio
    \newline
    misurabile e $X:\Omega\longrightarrow\mathcal{E}$ una variabile aleatoria, allora:
    \begin{center}
        $\mathbb{P}_X(\cdot)\coloneqq\mathbb{P}(X^{-1}(\cdot))$ al variare di $B\in\mathcal{E}$ 
    \end{center}
    è una \textbf{misura di probabilità}.
\end{theorem}

\vspace{5px}

\begin{itemize}
    \item $\mathbb{P}_X(\cdot)$ è detta \textbf{distribuzione di X} oppure \textbf{legge di X}
    \item Variabili aleatorie (V.A.) diverse che inducono la stessa distribuzione si dicono \textbf{identicamente distribuite}
\end{itemize}

\vspace{10px}

\begin{definition}
    Data $X$ una variabile aleatoria in $\mathbb{R}$, la funzione
    \newline 
    $F_X:\mathbb{R}\longrightarrow[0,1]$ t.c. $F_X(x)=\mathbb{P}(X\leq x)$ si dice \textbf{funzione di distribuzione}.
\end{definition}



Il seguente teorema ci fornirà importanti informazioni sulla natura della funzione di distribuzione.

\begin{theorem}
    Data una funzione di distribuzione $F_X$ valgono le seguenti:
    \begin{enumerate}
        \item $F_X$ è non decrescente
        \item $F_X$ è continua da destra
        \item  $\lim_{x\to+\infty} F_X(x) = 1$ e $\lim_{x\to-\infty} F_X(x) = 0$
    \end{enumerate}
\begin{proof}
\begin{enumerate}
    \item Siano $x,y\in\mathbb{R}$ t.c. $x\leq y$ allora:
    \begin{center}
        $\{\omega\in\Omega$ $|$ $X(\omega)\leq x\} \subseteq \{\omega\in\Omega$ $|$ $X(\omega)\leq y\}$
    \end{center}
    Dunque per monotonia di $\mathbb{P}$ si ha:
    \begin{center}
        $\mathbb{P}(X\leq x)\leq\mathbb{P}(X\leq y)$
        \hspace{4px}
        $\Rightarrow$
        \hspace{4px}
        $F_X(x)\leq F_X(y)$
    \end{center}
    \item Voglio mostrare $\lim_{x\to x_0^+} F_X(x) = F_X(x_0)$:
    \vspace{10px}
    \newline
    \noindent
    Considero $A_n=\{\omega\in\Omega$ $|$ $X(\omega)\in[x_0,x_0+\frac{1}{n})\}$ al variare di $n\in\mathbb{N}$ e osservo essere una successione decrescente all'insieme vuoto. 
    \vspace{10px}
    \newline
    \noindent
    Dunque applicando la definizione di funzione di probabilità:
    \begin{center}
        $\lim_{x\to x_0^+} F_X(x)=\lim_{n\to+\infty} F_X(x_0+\frac{1}{n})=\lim_{n\to+\infty}\mathbb{P}(X\leq x_0+\frac{1}{n})=$
        \vspace{5px}
        \newline
        $=\lim_{n\to+\infty}\mathbb{P}(\{\omega\in\Omega$ $|$ $X(\omega)\in(-\infty,x_0+\frac{1}{n})\})=$
        \vspace{5px}
        \newline
        $=\lim_{n\to+\infty}\mathbb{P}(\{\omega\in\Omega$ $|$ $X(\omega)\in(-\infty,x_0)\}\cup(A_n))=$
        \vspace{5px}
        \newline
        $=\mathbb{P}(X\leq x_0)+\lim_{n\to+\infty}\mathbb{P}(A_n)=F_X(x_0)$
    \end{center}
    \item 
    \begin{itemize}
        \item $\lim_{x\to+\infty} F_X(x)=\lim_{x\to+\infty}\mathbb{P}(\{\omega\in\Omega$ $|$ $X(\omega)\in(-\infty,x)\})=\mathbb{P}(\Omega)=1$
        \item $\lim_{x\to-\infty} F_X(x)=\lim_{x\to-\infty}\mathbb{P}(\{\omega\in\Omega$ $|$ $X(\omega)\in(-\infty,x)\})=\mathbb{P}(\emptyset)=0$
    \end{itemize}
\end{enumerate}
\end{proof}    
\end{theorem}



\begin{proposition}
Sia $X$ una variabile indipendente a valori in $\mathbb{R}$ e siano
\newline
$a,b\in\mathbb{R}$ tali che $a\leq b$, allora valgono:
\begin{enumerate}
    \item $\mathbb{P}(a< X\leq b) = F_X(b)-F_X(a) $
    \item $\mathbb{P}(X = a) = F_X(a) - \lim_{x\to a^-}F_X(x)$
    \item $\mathbb{P}(a< X< b) = \lim_{x\to b^-}F_X(x) - F_X(a)$
    \item $\mathbb{P}(X > a) = 1 - F_X(a)$
    \item $\mathbb{P}(a\leq X\leq b) = F_X(b) - \lim_{x\to a^-}F_X(x)$
    \item $\mathbb{P}(a< X) = \lim_{x\to a^-}F_X(x)$
    \item $\mathbb{P}(X\geq a) = 1 - \lim_{x\to a^-}F_X(x)$
\end{enumerate}
\begin{proof}
Si dimostra solo la 2. Le altre o sono conseguenze o seguono lo stesso schema dimostrativo.
Ricordiamo che dati $C,D\in\mathscr{F}$ tali che $C^c\cap D=\emptyset$ si ha la relazione:
\begin{center}
    $\mathbb{P}(C\setminus D)=\mathbb{P}(C)-\mathbb{P}(D)$
\end{center}
Dunque ponendo:
\begin{center}
    $C=\{\omega\in\Omega$ $|$ $X(\omega)\leq a\}$ e $D=\{\omega\in\Omega$ $|$ $X(\omega)\leq a-\frac{1}{n}\}$ 
\end{center}
Si ottiene così:
\begin{center}
    $\mathbb{P}(X=a)=\lim_{n\to+\infty}\mathbb{P}(C\setminus D)=\lim_{n\to+\infty}\mathbb{P}(C)-\mathbb{P}(D)=$
    \vspace{5px}
    \newline
    $=F_X(a)-\lim_{n\to+\infty}\mathbb{P}(X\leq a-\frac{1}{n})=F_X(a)-\lim_{x\to a^-}F_X(x)$
\end{center}
\end{proof}
\end{proposition}

\vspace{10px}
