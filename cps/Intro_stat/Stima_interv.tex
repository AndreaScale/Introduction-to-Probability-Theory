\section{Stima intervallare}

Sino ad ora abbiamo considerato solo ed esclusivamente la stima puntuale, dove i valori che vengono stimati sono punti. In questa sezioni ci occupiamo di affrontare la stima intervallare dove, crazy as it sounds, vengono stimati degli intervalli di valori, dunque ci chiediamo quando un parametro rientra in un interavllo.

Cominciamo con il dare la definizione fondamentale.

\begin{definition}
Sia $\overline{X}$ un c.c. di taglia $n$ estratto da una popolazione $X$ con densità $f(x;\Tt)$ e siano $T_1,T_2$ due statistiche tali che $\mathbb{P}(T_1<T_2)=1$. Considerato $\mathbb{P}(T_1<\Tt<T_2)=1-\alpha$, dove $1-\alpha$ non dipende da $\Tt$ e $\alpha\in(0,1)$ si definisce:
\begin{itemize}
    \item $[T_1,T_2]$ \textbf{intervallo di confidenza} di livello $(1-\alpha)$ per $\Tt$
    \item $(1-\alpha)$ \textbf{livello di confidenza}
\end{itemize}
Inoltre l'itervallo con estremi le realizzazioni delle statistiche è anch'esso detto itervallo di confidenza.
\end{definition}

\vspace{5px}
\noindent
Come esempio si consideri la seguente situazione:

Si prenda un c.c. di taglia 4 estratto da una normale di cui si conosce la varianza ($\sigma^2)$. Voglio determinare l'intervallo di confidenza per $\mu$ (il valore attesto) di livello $(1-\alpha)$. Per far ciò calcolo $a,b\in\mathbb{R}$ (i percentili) per cui $\mathbb{P}(a<Z<b)=1-\alpha$ dove $Z=\frac{\overline{X}_n-\mu}{\frac{\sigma}{\sqrt{n}}}\sim N(0,1)$. Dunque riconduco \newline $\mathbb{P}(a<Z<b)=1-\alpha$ a $\mathbb{P}(T_1<\mu<T_2)=1-\alpha$, ottenendo così gli entremi dell' itervallo di confidenza.

\vspace{10px}

\subsection{Metodo della quantità pivotale}

Un importante fattore che caratterizza gli stimatori intervallari è la loro dipendenza o meno dal parametro di cui cercano la stima. lo studio dei metodi pivotali ha proprio come oggetto stimatori \textit{non} dipendenti dal parametro.

\begin{definition}
$Q=Q(X_1,...,X_n;\Tt)$ funzione del c.c. e del paramtero $\Tt$ con funzione di distribuzione \textit{non} dipendente da $\Tt$ è detta \textbf{quantità pivotale}.
\end{definition}

Dunque se $X\sim f(x;\Tt)$ allora $Q(X,\Tt)$ ha la stessa distribuzione per ogni $\Tt$.

Due quantità pivotali abbastanza generali vengono individuate dal prossimo teorema.


\begin{theorem}
Dato un c.c. di taglia $n$ estratto da una popolazione con funzione di densità $f(\cdot;\Tt)$ e funzione di probabilità $F(x;\Tt)$ e continua rispetto a $x$, dunque: 
\begin{center}
    $\prod\limits_{i=1}^nF(X_i;\Tt)$ e $-\sum\limits_{i=1}^nlog(F(X_i;\Tt))$
\end{center}
Sono quantità pivotali.

\begin{proof}
Valuto la la funzione di probabilità della funzione della probabilità, che è ovviamente distribuita come una uniforme su $[0,1]$.

Dunque poniamo $Y=-log(F(X_i;\Tt))$ e ne studiamo la funzione di probabilità:
\[\mathbb{P}(Y\leq y)=1-\mathbb{P}(F(X_i;\Tt)<e^{-y})=\begin{cases} 1-e^{-y}, & 0<e^{-y}\leq1 \\ 0, & 1<e^{-y}
\end{cases}\]
Dunque osserviamo che $Y\sim Exp(1)$. Noi sappiamo inoltre che la somma di esponenziali con parametro $\lambda$ e indipendenti è una Gamma di parametri $(n,\lambda)$. Dunque: \[-\sum\limits_{i=1}^nlog(F(X_i;\Tt))\sim\Gamma(n,1)\]
Confermando così la tesi.

Per quanto riguarda la prima quantità mostriamo che può essere riscritta in funzione della seconda:
\[\mathbb{P}\bigg(\prod_{i=1}^nF(X_i;\Tt)\leq y\bigg)=\mathbb{P}\bigg(log\Big(\prod_{i=1}^nF(X_i;\Tt)\Big)\leq log(y)\bigg)=\]
\[\mathbb{P}\bigg(\sum\limits_{i=1}^nlog(F(X_i;\Tt))\leq log(y)\bigg)=1-\mathbb{P}\bigg(\sum\limits_{i=1}^n-log(F(X_i;\Tt))<-log(y)\bigg)\]
Dunque anche la prima quantità è pivotale.
\end{proof}
\end{theorem}